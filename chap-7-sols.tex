\documentclass{article}

%% Load additional packages
\usepackage[utf8]{inputenc}
\usepackage[none]{hyphenat}
\usepackage[backend=biber, dateabbrev=false]{biblatex}

\usepackage{amsmath}
\usepackage{amsfonts}
\usepackage{amssymb}
\usepackage{mathtools}
\usepackage{amsthm}

\usepackage{enumitem}
\usepackage{hyperref}
\usepackage{xcolor, framed}

\colorlet{shadecolor}{yellow}

%% Define additional commands
\theoremstyle{definition}
\newtheorem{exercise}{Exercise}
\newtheorem{lemma}{Lemma}[exercise]
\newtheorem{remark}{Remark}[exercise]

%% Set parameters
\setlength\parindent{0pt}

\title{
	\textbf{Kenneth Ireland \& Michael Rosen \\
		A Classical Introduction to Modern Number Theory \\
		(Chapter 7 Solutions)}
	\author{Khang Vinh Nguyen}
}
\date{\today}

\begin{document}

\maketitle

\newpage

\begin{exercise}
Use the method of Theorem 1 to show that a finite subgroup of the multiplicative group of a field is cyclic.
\end{exercise}
\begin{proof}
Let $G$ be a finite subgroup of $F^\times$ of order $n$. Then for any $d$ divides $n$, since $x^d = 1$ only has at most $d$ roots in a field, we know that $|\{ x \in G : x^d = 1\}| \leq d$. If there exists an element $\alpha$ of order $d$, then it generates a subset of $\{ x \in G : x^d = 1\}$ of cardinality $d$. From the previous inequality, we conclude that $\alpha$ generates the group $\{ x \in G : x^d = 1\}$. This group is obviously cyclic of order $d$ so there are exactly $\varphi(d)$ generators.
\\
\\
So either there are none, or there are $\varphi(d)$ elements in $G$ of order $d$. However, the order of $G$ is $n$, and each element has an order divide $n$, so
$$n \leq \sum_{d \mid n} \varphi(d) = n$$
In other words, the equality holds, so in particular, there are $\varphi(n)$ elements of order $n$. We conlcude that $G$ is cyclic.
\end{proof}

\newpage

\begin{exercise}
Let $\mathbb{R}$ and $\mathbb{C}$ be the real and complex numbers, respectively. Find the finite subgroups of $\mathbb{R}^*$ and $\mathbb{C}^*$ and show directiy that they are cyclic.
\end{exercise}
\begin{proof}
For $\mathbb{R}$, suppose $G$ is a finite subgroup of $\mathbb{R}^*$. Then every element of $G$ has finite order. However, the only non-zero real numbers of finite order are $\{ \pm 1 \}$, so either $G = \{ 1 \}$ (trivially cyclic) or $G = \{ \pm 1 \}$ (generates by $-1$).
\\
\\
*For $\mathbb{C}$, suppose $G$ is a finite subgroup of $\mathbb{C}^*$. Then again, every element of $G$ has finite order. Consider $z^n = 1$, then we know that $e^{\frac{2 \pi i k}{n}}$ for integers $0 \leq k < n$ are all non-zero complex numbers of finite order. Pick $e^{\frac{2 \pi i k}{n}} \in G$ so that $\gcd(k, n) = 1$ and $n$ is largest (as $G$ only has finitely many elements). Then evert $n$-th root of unity must be in $G$, and we will show those are all element in $G$.
\\
\\
Let $e^{\frac{2 \pi i l}{d}}$ with $\gcd(l, d) = 1$ be a number in $G$, then $d \leq n$ by maximality. If $d \nmid n$, then there exists some prime divisor $p$ of $d$ that does not divide $n$. Multiplying if necessary, we should get $e^{\frac{2 \pi i}{p}} \in G$ and similarly, $e^{\frac{2 \pi i}{n}} \in G$. But since
$$\frac{1}{p} + \frac{1}{n} = \frac{p + n}{pn}$$
is at reduced form (indeed, if $d \mid p + n, pn$, then $d \mid p^2$, and so $d = 1$ since $p \nmid n$), we have a contradiction to maximality of $n$.
\\
\\
In other words, if $e^{\frac{2 \pi i l}{d}} \in G$ (with $\gcd(l, d) = 1$), then $d \mid n$. We conclude that $G$ is cyclic of order $n$.
\end{proof}

\newpage

\begin{exercise} \label{c7-ex-3}
Let $F$ be a field with $q$ elements and suppose that $q \equiv 1 \pmod{n}$. Show that for $\alpha \in F^*$, the equation $x^n = \alpha$ has either no solutions or $n$ solutions.
\end{exercise}
\begin{proof}
We show that if it has solution, then it has $n$ solution. Note that the equation $x^n = 1$ has exactly $n$ solution in $F^*$ since $n$ divides the order of $F^*$, which is $q - 1$ (also because $F^*$ is cyclic). Thus given some $t^n = \alpha$, we have $(tz)^n = \alpha$ for each $z^n = 1$. Since neither $t$ nor $z$ is zero, we get $n$ distinct solutions. However the polynomial $x^n - \alpha$ has at most $n$ solution, as $F$ is an integral domain, so we conclude that $x^n = \alpha$ has $n$ solutions.
\end{proof}
\begin{exercise}[continuation]
Show that the set of $\alpha \in F^*$ such that $x^n = \alpha$ is solvable is a subgroup with $(q - 1)/n$ elements.
\end{exercise}
\begin{proof}
Given $\alpha = t^n$ and $\beta = s^n$ for some $t, s \in F^*$, then we have $\alpha \beta^{-1} = (ts^{-1})^n$, so such set $S$ is a subgroup of $F^*$. Define the following map
$$F^* \to F^*$$
$$x \mapsto x^n$$
We know that the image of the map is $S$, while each non-empty fiber (where $x^n = \alpha$ is solvable) has exactly $n$ elements (\hyperref[c7-ex-3]{Exercise \ref*{c7-ex-3}}). Thus, $|F^*| = n |S|$, or equivalently, $|S| = \frac{q - 1}{n}$.
\end{proof}
\begin{exercise}[continuation] \label{c7-ex-5}
Let $K$ be a field containing $F$ such that $[K : F] = n$. For all $\alpha \in F^*$, show that the equation $x^n = \alpha$ has $n$ solutions in $K$. [Hint: Show that $q^n-1$ is divisible by $n(q - 1)$ and use the fact that $\alpha^{q - 1} = 1$.]
\end{exercise}
\begin{proof}
If $x^n = \alpha$ has $n$ solutions in $K$ for each $\alpha \in F^*$, each such satisfies $x^{n (q - 1)} = \alpha^{q - 1} = 1$, so $\{ x \in K^* : x^{n (q - 1)} = 1 \}$ has exactly $n (q - 1)$ elements. This should tell us to show that $q^n - 1$ (the order of $K^*$) is divisible by $n (q - 1)$.
\\
\\
Indeed, still assume $q \equiv 1 \pmod{n}$, we then get $q^{n - 1} + q^{n - 2} + \cdots + 1 \equiv n \equiv 0 \pmod{n}$, so $(q - 1)n \mid (q - 1) (q^{n - 1} + q^{n - 2} + \cdots + 1) = q^n - 1$. But $K^*$ is cylic, so there is a subgroup of order $n (q - 1)$. In other words, $x^{n (q - 1)} = 1$ has exactly $n (q - 1)$ solutions in $K$.
\\
\\
*Since $F^* \subseteq K^*$, every equation $x^n = \alpha$ either has no solution or $n$ solutions in $K$ (\hyperref[c7-ex-3]{Exercise \ref*{c7-ex-3}}), together with the fact that $n \mid n (q - 1) \mid q^n - 1$). Each such solution for some $\alpha \in F^*$, will lead to a solution of $x^{n(q - 1)} = 1$, since $x^{n(q - 1)} = \alpha^{q - 1} = 1$ (where $|F^*| = q - 1$). Vice versa, if $x^{n (q - 1)} = 1$, then $(x^n)^{q - 1} = 1$, so $x^n \in F^*$.
\\
\\
Now let $C_\alpha \in \{ 0, n \}$ be the number of solutions of $x^n = \alpha$ in $K$, we get
$$n (q - 1) \leq \sum_{\alpha \in F^*} C_\alpha \leq n |F^*| = n (q - 1)$$
Equality happens, and so $C_\alpha = n$.
\end{proof}

\newpage

\begin{exercise} \label{c7-ex-6}
Let $K \supseteq F$ be finite fields with $[K : F] = 3$. Show that if $\alpha \in F$ is not a square in $F$, it is
not a square in $K$.
\end{exercise}
\begin{proof}
Note that $x^2 - \alpha$ is irreducible over $F$ since it is quadratic, and there is no root of it in $F$. Thus, the splitting field $H$ is a degree $2$ extension of $F$. If $K$ contains a square root of $\alpha$, then $x^2 - \alpha$ splits in $K$, so $H$ must be a subfield of $K$. Yet
$$3 = [K:F] = [K : H] [H : F] = 2 [K : H]$$
which is impossible
\end{proof}
\begin{exercise}
Generalize \hyperref[c7-ex-6]{Exericse \ref*{c7-ex-6}} by showing that if $\alpha$ is not a square in $F$, it is not a square in any extension of odd degree and is a square in every extension of even degree.
\end{exercise}
\begin{proof}
The odd degree is clear since it cannot equal to any even number. As for even degree, suppose $[K : F] = 2n$. Then if $|F| = q$, we get $|K| = q^{2n}$. Note that $x^2 - \alpha$ is monic and irreducible over $F$, and $2 \mid 2n$, so $x^2 - \alpha \mid x^{q^{2n}} - x$. But $x^{q^{2n}} - x$ splits in $K$, so $x^2 - \alpha$ also splits completely in $K$, and so $\alpha$ is a square in $K$.
\end{proof}

\newpage

\begin{exercise}
In a field with $2^n$ elements what is the subgroup of squares?
\end{exercise}
\begin{proof}
Consider the Frobenius map on $F$
$$x \mapsto x^2$$
It is field homomorphism since $(x + y)^2 = x^2 + 2xy + y^2 = x^2 + y^2$, and $(xy)^2 = x^2 y^2$. It is also injective since $x^2 = y^2$ iff $x^2 - y^2 = (x - y)^2 = 0$ iff $x = y$, so since $F$ is finite, the map is a also surjective. Since $0$ maps to $0$, the rest maps to themselves. In other word, the subgroup of squares is exactly the multiplicative group $F^*$.
\end{proof}

\newpage

\begin{exercise}
If $K \supseteq F$ are finite fields, $|F| = q, \alpha \in F, q \equiv 1 \pmod{n}$, and $x^n = \alpha$ is not solvable in $F$, show that $x^n = \alpha$ is not solvable in $K$ if $\gcd(n, [K : F]) = 1$.
\end{exercise}
\begin{proof}
We show that $x^n = \alpha$ has a solution in a field $F$ of $q$ elements iff $\alpha^{\frac{q - 1}{n}} = 1$ (assuming $\alpha \neq 0$, and $n \mid q - 1$). Suppose $x_0^n = \alpha$ for some $x_0 \in F$, then $\alpha^{\frac{q - 1}{n}} = x_0^{q - 1} = 1$. Vice versa, suppose $\alpha^{\frac{q - 1}{n}} = 1$, then $\alpha$ is an element of the subgroup of order $\frac{q - 1}{n}$ in $K^*$. But if $g$ generates $K^*$, then $g^n$ generates the subgroup of order $\frac{q - 1}{n}$, so $g^{nk} = \alpha$ for some $k \leq \frac{q - 1}{n}$. Letting $x = g^k$, then we get a solution to $x^n = \alpha$.
\\
\\
Thus, suppose $x^n = \alpha$ is solvable in $K$. Then $\alpha^{\frac{q^m - 1}{n}} = 1$, where $|K| = q^m$ (and so $[K : F] = m$). However, $x^n = \alpha$ is not solvable in $F$, so $\beta = \alpha^{\frac{q - 1}{n}} \neq 1$. Now note that $\beta^{q^{m - 1} + \cdots + q + 1} = \alpha^{\frac{q^m - 1}{n}} = 1$. On the other hand, $\beta^n = \alpha^{q - 1} = 1$ (since $\alpha \in F$, and $\alpha \neq 0$ due to $x^n = \alpha$ is not solvable in $F$).
\\
\\
We then must have $\gcd(q^{m - 1} + \cdots + q + 1, n) > 1$, as $\beta \neq 1$. But $q \equiv 1 \pmod{n}$, so $q^{m - 1} + \cdots + q + 1 \equiv m \pmod{n}$. From there, $\gcd(q^{m - 1} + \cdots + q + 1, n) = \gcd(m, n) > 1$. By contraposition, we have Q.E.D.
\end{proof}

\newpage

\begin{exercise} \label{c7-ex-10}
Let $K \supseteq F$ be finite fields and $[K : F] = 2$ (with $|F| = q$). For $\beta \in K$ show that $\beta^{1 + q} \in F$ and moreover that every element in $F$ is of the form $\beta^{1 + q}$ for some $\beta \in K$.
\end{exercise}
\begin{proof}
If $\beta = 0$, then $\beta^{1 + q} = 0 \in F$. Otherwise, $\beta^{q^2 - 1} = 1$ (since $|K| = q^2$), so $(\beta^{q + 1})^{q - 1} = 1$. But any $\alpha^{q - 1} = 1$ must imply that $\alpha \in F$, so we have $\beta^{q + 1} \in F$. On the other hand, by \hyperref[c7-ex-3]{Exercise \ref*{c7-ex-3}}, the equation $x^{q + 1} = \alpha \in F^*$ has either no solution or $q + 1$ solutions in $K$ (as $q + 1 \mid q^2 - 1$).
\\
\\
Denoting $C_\alpha \in \{ 0, q + 1 \}$ to be the solutions of $x^{q + 1} = \alpha$, we get
$$q^2 - 1 = \sum_{\alpha \in F^*} C_\alpha \leq (q + 1) |F^*| = q^2 - 1$$
Thus, $C_\alpha = q + 1$ for all $\alpha \in F^*$.
\end{proof}
\begin{exercise}
With the situation being that of \hyperref[c7-ex-10]{Exercise \ref*{c7-ex-10}} suppose that $\alpha \in F$ has order $q - 1$. Show that there is a $\beta \in K$ with order $q^2 - 1$ such that $\beta^{1 + q} = \alpha$.
\end{exercise}
This is just purely group theory, so we prove the following lemmas
\begin{lemma}
Given $\gcd(k, d) = 1$ and $d \mid n$ (with $d \neq 0$), there exists some integer $l$ such that $k \equiv l \pmod{d}$ and $\gcd(l, n) = 1$
\end{lemma}
\begin{proof}
Let $p_1, p_2, \hdots, p_r$ be all prime factors of $n$ that is not a factor of $d$. By Chinese remainder theorem, there exists some $l$ satisfying following congruences
$$l \equiv k \pmod{d} \mbox{ and } l \equiv 1 \pmod{p_i}, 1 \leq i \leq r$$
For each prime divisor $q$ of $n$, if $q \mid d$, then $\gcd(l, q) = \gcd(k, q) = 1$ since $\gcd(k, d) = 1$. Otherwise, $q \nmid d$, so it must be some $p_i$ above. But since $l \equiv 1 \pmod{p_i}$, we have $\gcd(l, q) = \gcd(1, p_i) = 1$. Therefore, $l$ is not divisible by any prime factor of $n$, so $\gcd(l, n) = 1$.
\end{proof}
\begin{lemma}
Given $G$ is cylic group of order $n$, and $\alpha$ generates a subgroup of order $d$, then there exists a generator $\beta$ of $G$ such that $\alpha = \beta^{n/d}$.
\end{lemma}
\begin{proof}
Let $g$ be a generator of $G$. Since there is only one subgroup of order $d$ in $G$, we know that $g^{(n/d) k} = \alpha$ for some $0 \leq k \leq d - 1$. Note that $\alpha = g^{(n/d) k}$ has order $d$, $k$ must be relatively prime to $d$. By the previous lemma, there must be some $l \equiv k \pmod{d}$ such that $\gcd(l, n) = 1$. We then denote $\beta = g^l$, which is of order $n$. On the other hand, $k \equiv l \pmod{d}$, so $(n/d) k \equiv (n/d) l \pmod{n}$, so we have $\beta^{n/d} = g^{(n/d)l} = g^{(n/d)k} = \alpha$.
\end{proof}
\begin{proof}[Solution to the original]
By the hypothesis, $\alpha$ still has order $q - 1$ in $K^*$. Since $K^*$ is cyclic of order $q^2 - 1$, the previous lemma should give us some $\beta$ of order $q^2 - 1$ such that
$$\beta^{q + 1} = \beta^{\frac{q^2 - 1}{q - 1}} = \alpha$$
\end{proof}

\newpage
 
\begin{exercise} \label{c7-ex-12}
Use Proposition 7.2.1 to show that given a field $k$ and a polynomial $f(x) \in k[x]$ there is a field $K \supseteq k$ such that $[K : k]$ is finite and $f(x) = (x - \alpha_1) (x - \alpha_2) \cdots (x - \alpha_n)$ in $K[x]$.
\end{exercise}
\begin{proof}
We use induction on the degree of $f(x)$. If $\deg f(x) = 1$, then $K = k$ should be suffice. Assuming, $\deg f(x) = k + 1$, then there exists a field $K$ containing $k$ such that $f(\alpha) = 0$ for some $\alpha \in K$. Since $f(x)$ can be lift to a polynomial over $K$, we can factor $(x - \alpha)$ out, so that there exists some $g(x) \in K[x]$ such that $f(x) = (x - \alpha) g(x)$. Since $\deg g(x) = k$, we can find another field $L$ containing $K$ such that $g(x) = (x - \beta_1) (x - \beta_2) \cdots (x - \beta_k)$ in $K$ by induction.
\end{proof}

\newpage

\setcounter{exercise}{13}
\begin{exercise}
Let $F$ be a field with $q$ elements and $n$ a positive integer. Show that there exist irreducible polynomials in $F[x]$ of degree $n$.
\end{exercise}
\begin{proof}
We know that there is a field $K$ containing $F$ such that $K$ has $q^n$ elements. Let $\alpha$ be a generator of $K^*$, then $K = F(\alpha)$. If $f(x)$ is the minimal polynomial of $\alpha$ over $F$, then $n = [K : F] = [F(\alpha) : F] = \deg f(x)$. In other words, $f(x)$ is an irreducible polynomial of degree $n$.
\end{proof}

\newpage

\begin{exercise}
Let $x^n - 1 \in F[x]$, where $F$ is a finite field with $q$ elements. Suppose that $\gcd(q, n) = 1$. Show that $x^n - 1$ splits into linear factors in some extension field and that the least degree of such a field is the smallest integer $f$ such that $q^f \equiv 1 \pmod{n}$.
\end{exercise}
\begin{proof}
Let $K$ be the splitting field of $x^n - 1$ over $F$. Suppose $|K| = q^f$, then $x^n - 1 \mid x^{q^f} - x = (x^{q^f - 1} - 1) x$. Since $\gcd(x^n - 1, x) = 1$, we know that $x^n - 1 \mid x^{q^f - 1} - 1$. Now consider the subset $G = \{ x \in K : x^n = 1 \}$ of $K^*$, which has $n$ elements: one can very that $G$ is a group, so the order of $G$ divides $K^*$. In other words, $n \mid q^f - 1$.
\\
\\
We now show that $f$ is the least such positive integer. Indeed, since $\gcd(q, n) = 1$, let $e$ be the order of $q$ modulo $n$. Then $n \mid q^e - 1$, so consider the finite field $L$ of $q^e$ elements: it is an extension of $F$ (since $F$ has $q$ elements), and there is a subgroup of $L$ of order $n$ (note $L^*$ is cyclic). Thus, $x^n - 1$ splits in $L$. By minimality, $K$ must be a subfield of $L$, which shows that $f \leq e$. Since $q^f \equiv 1 \pmod{n}$, we have $e = f$ (by definition of multiplicative order modulo $n$).
\end{proof}

\newpage

\begin{exercise}
Calculate the monic irreducible polynomials of degree $4$ in $\mathbb{Z}/2 \mathbb{Z} [x]$.
\end{exercise}
\begin{proof}
Such polynomial cannot have $0$ as root, so the constant coefficient must be $1$. Also, such polynomial cannot have $1$ as root, so the sum of coefficients (which are either $0$ or $1$) must be odd. In other words, there are odd number of terms in the polynomial, which left us with 4 possibilities
$$x^4 + x^3 + 1, x^4 + x^2 + 1, x^4 + x + 1, x^4 + x^3 + x^2 + x + 1$$
\\
Now some of them may factor into 2 quadratics, neither of which has roots. But the only quadratic polynomial over $\mathbb{Z}/2 \mathbb{Z}$ that has no root is $x^2 + x + 1$, so such polynomial must be $(x^2 + x + 1)^2 = x^4 + x^2 + 1$. Thus, this eliminates one polynomial among the previous 4, and we get 3 irreducible polynomials
$$x^4 + x^3 + 1, x^4 + x + 1, x^4 + x^3 + x^2 + x + 1$$
\end{proof}

\newpage

\begin{exercise}
Let $q$ and $p$ be distinct odd primes. Show that the number of monic irreducibles of degree $q$ in $\mathbb{Z}/p \mathbb{Z} [x]$ is $q^{-1} (p^q - p)$.
\end{exercise}
\begin{proof}
Since the number of monic irreducible polynomails of degree $n$ is
$$N_n = \frac{1}{n} \sum_{d \mid n} \mu \left( \frac{n}{d} \right) p^d$$
Substitute $n = q$, and we get $N_q  = \frac{1}{q} (\mu(q) p + \mu(1) p^q) = \frac{1}{q} (p^q - p)$.
\end{proof}

\newpage

\begin{exercise}
Let $p$ be a prime with $p \equiv 3 \pmod{4}$. Show that the residue classes modulo $p$ in $\mathbb{Z}[i]$ form a field with $p^2$ elements.
\end{exercise}
\begin{proof}
By assumption, $p$ is a prime in $\mathbb{Z}[i]$, and since $\mathbb{Z}[i]$ is Euclidean, it implies that $(p)$ is maximal, thus $\mathbb{Z}[i] / (p)$ is a finite field.
\\
\\
Now note that for a given integer $n$, $a + bi \equiv c + di \pmod{n}$ if there is some $r + si \in \mathbb{Z}[i]$ such that $n (r + si) = (a + bi) - (c + di)$. Equivalently, $a - c = nr$ and $b - d = ns$, so $a$ must be congruent to $c$ and $b$ must be congruent to $d$ modulo $n$. Thus, $\mathbb{Z}[i] / (n) = \{ a + bi + (n) : 0 \leq a < n, 0 \leq b < n \}$, of which there are exactly $n^2$ distinct elements. Substituting $n = p$ should get us a finite field $\mathbb{Z}[i] / (p)$ of $p^2$ elements.
\end{proof}

\newpage

\begin{exercise} \label{c7-ex-19}
Let $F$ be a finite field with $q$ elements. If $f(x) \in F[x]$ has degree $t$, put $|f| = q^t$. Verify the formal identity $\sum_f |f|^{-s} = (1 - q^{1 - s})^{-1}$. The sum is over all monic polynomials.
\end{exercise}
\begin{proof}
By counting the number of ways each coefficient (except leading coefficient, since we require a polynomial to be monic) can take in $F$, we have $q^t$ monic polynomials of degree $t$ over $F$. Hence
$$\sum_f |f|^{-s} = \sum_{t = 0}^\infty q^t \cdot (q^t)^{-s} = \sum_{t = 0}^\infty (q^{1 - s})^t = \frac{1}{1 - q^{1 - s}}$$
\end{proof}

\newpage

\begin{exercise}
With the notation of \hyperref[c7-ex-19]{Exereise \ref*{c7-ex-19}}, let $d(f)$ be the number of monic divisors of $f$ and $\sigma(f) = \sum_{g \mid f} |g|$ where the sum is over the monic divisors of $f$. Verify the following identities:
\begin{enumerate}
	\item $\sum_f d(f) |f|^{-s} = (1 - q^{1 - s})^{-2}$.
	\item $\sum_f \sigma(f) |f|^{-s} = (1 - q^{1 - s})^{-1} (1 - q^{2 - s})^{-1}$.
\end{enumerate}
\end{exercise}
\begin{proof}
We use the idea from Dirichlet series as follows: if $\alpha, \beta: \mathcal{M} \to \mathbb{C}$ are maps from the set of monic polynomials $\mathcal{M}$ over (a finite field) $F$ to $\mathbb{C}$, then
\begin{align*}
\left( \sum_{f} \frac{\alpha(f)}{|f|^s} \right) \left( \sum_{g} \frac{\beta(g)}{|g|^s} \right) & = \sum_{f, g} \frac{\alpha(f) \beta(g)}{|f|^s |g|^s} = \sum_{f, g} \frac{\alpha(f) \beta(g)}{q^{s (\deg f + \deg g)}} \\
& = \sum_{f, g} \frac{\alpha(f) \beta(g)}{q^{s \deg (fg)}} = \sum_{f, g} \frac{\alpha(f) \beta(g)}{|fg|^s} \\
& = \sum_h \frac{1}{|h|^s} \sum_{fg = h} \alpha(f) \beta(g) \\
& = \sum_h \frac{(\alpha * \beta)(h)}{|h|^s}
\end{align*}
where $(\alpha * \beta)(h) = \sum_{fg = h} \alpha(f) \beta(g) = \sum_{f \mid h} \alpha(f) \beta \left( \frac{h}{f}) \right)$ is the Dirichlet convolution of $\alpha$ and $\beta$.
\\
\\
\underline{Part 1:} since $d(f) = \sum_{g \mid f} 1 = (1 * 1) (f)$ (where $1(f) = 1$ for all $f$ monic), so
$$\sum_f d(f) |f|^{-s} = \left( \sum_f |f|^{-s} \right)^2 = \frac{1}{(1 - q^{1 - s})^2}$$
\underline{Part 2:} since $\sigma(f) = \sum_{g \mid f} |g|$ is the convolution of the maps $1(g)$ and $|g|$, we get
\begin{align*}
\sum_f \sigma(f) |f|^{-s} & = \left( \sum_f |f|^{-s} \right) \left( \sum_f |f| \cdot |f|^{-s} \right) \\
& = \left( \sum_f |f|^{-s} \right) \left( \sum_f |f|^{-(s - 1)} \right) \\
& = \frac{1}{(1 - q^{1 - s}) (1 - q^{1 - (s - 1)})} \\
& = \frac{1}{(1 - q^{1 - s}) (1 - q^{2 - s})}
\end{align*}
\end{proof}

\newpage

\begin{exercise}
Let $F$ be a field with $q = p^n$ elements. For $\alpha \in F$ set $f(x) = (x - \alpha) (x - \alpha^p) (x - \alpha^{p^2}) \cdots (x - \alpha^{p^{n - 1}})$. Show that $f(x) \in \mathbb{Z}/p \mathbb{Z}[x]$. In particular, $\alpha + \alpha^p + \cdots + \alpha^{p^{n - 1}}$ and $\alpha \cdot \alpha^p \cdots \alpha^{p^{n - 1}}$ are in $\mathbb{Z}/p \mathbb{Z}$.
\end{exercise}
\begin{proof}
Consider the Frobenius automorphis $\tau: F \to F$ by sending $\tau(r) = r^p$ for each $r \in F$. It is a field automorphism since
\begin{enumerate}
	\item $(r + s)^p = r^p \sum_{k = 1}^{p - 1} {p \choose k} r^{p - k} s^k + s^p = r^p + s^p$
	\item $(rs)^p = r^p s^p$.
	\item If $r^p = s^p$, then $(r - s)^p = r^p - s^p = 0$ (it works in charateristic $2$, as $(r - s)^2 = r^2 + s^2 = r^2 - s^2$), thus $r = p$. In other words, $\tau$ is injective.
	\item Since $F$ is finite, an injective map is bijective.
\end{enumerate}
We the lift $\tau$ to the polynomials, which should be a ring homomorphism. Note that the fixed points $r^p = r$ of $F$ is exactly elements of the prime subfield $\mathbb{Z}/p \mathbb{Z}$, and so the polynomials fixed by $\tau$ is exactly the ones with coefficient in $\mathbb{Z}/p \mathbb{Z}$.
\\
\\
From there, we just need to note that
\begin{align*}
\tau(f(x)) & = (x - \alpha^p) (x - \alpha^{p^2}) \cdots (x - \alpha^{p^{n - 1}}) (x - \alpha^{p^n}) \\
& = (x - \alpha^p) (x - \alpha^{p^2}) \cdots (x - \alpha^{p^{n - 1}}) (x - \alpha) \\
& = f(x)
\end{align*}
\end{proof}

\newpage

\begin{exercise}[continuation]
Set $\operatorname{tr}(\alpha) = \alpha + \alpha^p + \cdots + \alpha^{p^{n - 1}}$. Prove that
\begin{enumerate}
	\item $\operatorname{tr}(\alpha) + \operatorname{tr}(\beta) = \operatorname{tr}(\alpha + \beta)$.
	\item $\operatorname{tr}(a \alpha) = a \operatorname{tr}(\alpha)$ for $a \in \mathbb{Z}/p\mathbb{Z}$.
	\item There is an $\alpha \in F$ such that $\operatorname{tr}(\alpha) \neq 0$.
\end{enumerate}
\end{exercise}
\begin{proof} \ \\
\underline{Part 1:} since we have shown $(\alpha + \beta)^p = \alpha^p + \beta^p$, we can prove that
$$(\alpha + \beta)^{p^k} = (\alpha^p + \beta^p)^{p^{k - 1}} = (\alpha^{p^2} + \beta^{p^2})^{p^{k - 2}} = \alpha^{p^k} + \beta^{p^k}$$
for any $k \geq 0$. Thus,
\begin{align*}
\operatorname{tr}(\alpha) + \operatorname{tr}(\beta) & = (\alpha + \beta) + (\alpha^p + \beta^p) + \cdots + (\alpha^{p^{n - 1}} + \beta^{p^{n - 1}}) \\
& = (\alpha + \beta) + (\alpha + \beta)^p + \cdots + (\alpha + \beta)^{p^{n - 1}} \\
& = \operatorname{tr}(\alpha + \beta)
\end{align*}
\underline{Part 2:} since $a^p = a$ for all $a \in \mathbb{Z}/p\mathbb{Z}$, we get
$$\operatorname{tr}(a \alpha) = a \alpha + a \alpha^p + \cdots + a \alpha^{p^{n - 1}}) = a \operatorname{tr}(\alpha)$$
\underline{Part 3:} if suppose $\operatorname{tr}(\alpha) = 0$ for all $\alpha \in F$, then the polynomial $x + x^p + \cdots + x^{p^{n - 1}}$ has a total of $p^n$ in $F$. However, the degree is less than $p^n$, a contradiction.
\end{proof}

\newpage

\begin{exercise}[continuation]
For $\alpha \in F$ consider the polynomial $x^p - x - \alpha \in F[x]$. Show that this polynomial is either irreducible or the product of linear factors. Prove that the latter alternative holds iff $\operatorname{tr}(\alpha) = 0$.
\end{exercise}
\begin{proof}
Note that $(x + 1)^p - (x + 1) - \alpha = x^p + 1 - x - 1 - \alpha = x^p - x - \alpha$, so if it has a root in $F$, then it all of its roots are in $F$ (by shifting a root).
\\
\\
Let $K$ be the splitting field of the polynomial (possibly equal to $F$), and say $r \in K$ be one of its root. For each root $r + a$ ($a \in \mathbb{Z}/p\mathbb{Z}$), let $m_a (x)$ be its minimal (monic) polynomial. Since $m_a (x + a - b)$ has root $r + b$, we know that $\deg m_a (x) \leq \deg m_b (x)$. By symmetry, we must have $\deg m_a (x) = \deg m_b (x) = m$ for all $a, b \in \mathbb{Z}/p\mathbb{Z}$. On the other hand, by minimality, we know that $m_a (x) \mid x^p - x - \alpha$, so $x^p - x - \alpha$ is the product of \textit{distinct} $m_a$. Suppose there are $k$ such, then $p = km$. Hence, either
\begin{enumerate}
	\item $m = 1$, which implies $x^p - x - a$ splits completely.
	\item $m = p$, which implies $x^p - x - \alpha$ is irreducible (since a minimal polynomial is irreducible).
\end{enumerate}
\ \\
*Suppose $x^p - x - \alpha$ splits completely in $F[x]$. Equivalently, there exists some $r \in F$ such that $x^p - x - \alpha = (x - r) [x - (r + 1)] [x - (r + 2)] \cdots [x - (r + p - 1)]$. Consider the constant coefficient:
\begin{align*}
\operatorname{tr}(\alpha) & = \sum_{k = 0}^{n - 1} (r^p - r)^{p^k} = \sum_{k = 0}^{n - 1} r^{p^{k + 1}} - r^{p^k} = r^{p^n} - r = 0
\end{align*}
Vice versa, suppose $x^p - x - \alpha$ is irreducible over $F$. This means that $r^p - r \neq \alpha$ for all $r \in F$. Since the Frobenius automorphism $\tau$ has $\mathbb{Z}/p\mathbb{Z}$ as its set of fixed point, $\tau - 1: F \to F$ is a group homomorphism with kernel $\mathbb{Z}/p\mathbb{Z}$, thus its image $S = \{ r^p - r : r \in F \}$ must have cardinality of $p^{n - 1}$ (using 1st isomorphism theorem).
\\
\\
On the other hand, we already prove that $\operatorname{tr}(\gamma) = 0$ for all $\gamma \in S$. So the elements of $S$ should be roots of $x + x^p + \cdots + x^{p^{n - 1}} = 0$. However, $|S| = p^{n - 1}$, so $S$ is exactly the set of roots of $x + x^p + \cdots + x^{p^{n - 1}} = 0$. Equivalently, $S$ is exactly the set of $\gamma$ such that $\operatorname{tr}(\gamma) = 0$. Since $x^p - x - \alpha$ is irreducible, we have $\alpha \notin S$, and so $\operatorname{tr}(\alpha) \neq 0$.
\end{proof}

\newpage

\begin{exercise}
Suppose that $f(x) \in \mathbb{Z}/p\mathbb{Z}[x]$ has the property that $f(x + y) = f(x) + f(y) \in \mathbb{Z}/p \mathbb{Z}[x, y]$. Show that $f(x)$ must be of the form $a_0 x + a_1 x^p + a_2 x^{p^2} + \cdots + a_m x^{p^m}$.
\end{exercise}
\begin{proof}
Using Lucas theorem, we know that, unless $n$ is a power of $p$, then there is some $1 \leq k \leq n-1$ such that ${n \choose k} \not\equiv 0 \pmod{p}$. Indeed, if $n$ is not a power of $p$, then either $a_r > 1$, or $a_i \neq 0$ for some $1 \leq i \leq r - 1$ where $n = a_r p^r + \cdots + a_1 p + a_0$ is the representation of $n$ in base $p$.
\begin{enumerate}
	\item If $a_r > 1$, pick $k = p^r < a_r p^r \leq n$ . Then
	$${n \choose k} \equiv {a_r \choose 1} {a_{r - 1} \choose 0} {a_{r - 2} \choose 0} \cdots {a_0 \choose 0} \equiv a_r \not\equiv 0 \pmod{p}$$
	\item If $a_i \geq 1$ for some $1 \leq i \leq r - 1$, pick $k = p^i \leq a_i p^i < a_i p^i + p^r \leq n$. Then
	$${n \choose k} \equiv {a_r \choose 0} \cdots {a_{i + 1} \choose 0} {a_i \choose 1} {a_{i - 1} \choose 0} \cdots {a_0 \choose 0} \equiv a_i \not\equiv 0 \pmod{p}$$
\end{enumerate}
So if $f(x)$ contains some non-zero term $a_n x^n$ such that $n$ is not a power of $p$, then $f(x + y)$ has non-zero term $a_n {n \choose k} x^k y^{n - k}$ modulo $p$, while $f(x) + f(y)$ only has terms of the form $x^i$ or $y^j$, a contradiction to the hypothesis. As for the constant term $a_0$, we have $a_0 = 2 a_0$, so $a_0 = 0$.
\end{proof}

\end{document}