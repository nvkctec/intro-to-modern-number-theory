\documentclass{article}

%% Load additional packages
\usepackage[utf8]{inputenc}
\usepackage[none]{hyphenat}
\usepackage[backend=biber, dateabbrev=false]{biblatex}

\usepackage{amsmath}
\usepackage{amsfonts}
\usepackage{amssymb}
\usepackage{mathtools}
\usepackage{amsthm}

\usepackage{enumitem}
\usepackage{hyperref}
\usepackage{xcolor, framed}

\colorlet{shadecolor}{yellow}

%% Define additional commands
\theoremstyle{definition}
\newtheorem{exercise}{Exercise}
\newtheorem{remark}{Remark}[exercise]

%% Set parameters
\setlength\parindent{0pt}

\title{
	\textbf{Kenneth Ireland \& Michael Rosen \\
		A Classical Introduction to Modern Number Theory \\
		(Chapter 1 Solutions)}
	\author{Khang Vinh Nguyen}
}
\date{\today}

\begin{document}

\maketitle

\newpage

\begin{exercise} \label{c1-ex-1}
Let $a$ and $b$ be nonzero integers. We can find nonzero integers $q$ and $r$ such that $a = qb + r$, where $0 \leq r < b$. Prove that $(a, b) = (b, r)$ (these are ideals in $\mathbb{Z}$).
\end{exercise}
\begin{proof}
Let $m \in (a, b)$, then $m = ax + by$ for some $x, y \in \mathbb{Z}$. But $a = qb + r$, so we get $m = (qb + r) x + by = r x + b (qx + y)$, so $m \in (b, r)$ by definition. Vice versa, if $m \in (b, r)$, then $m = bx + ry$ for some $x, y \in \mathbb{Z}$. Using $a = qb + r$, we get $m = bx + (a - qb) y = ay + b(x - qy)$, so $m \in (a, b)$. We conclude that $(a, b) = (b, r)$ as sets.
\end{proof}

\newpage

\begin{exercise}[continuation] \label{c1-ex-2}
If $r \neq 0$, we can find $q_1$ and $r_1$ such that $b = q_1 r + r_1$, with $0 \leq r_1 < r$. Show that $\gcd(a, b) = \gcd(r, r_1)$. This process can be repeated. Show that it must end in finitely many steps. Show that the last nonzero remainder must equal $\gcd(a, b)$. The process looks like
\begin{align*}
a & = qb + r, & 0 & \leq r < b, \\
b & = q_1 r + r_1, & 0 & \leq r_1 < r, \\
r & = q_2 r_1 + r_2, & 0 & \leq r_2 < r_1, \\
& \enspace\vdots & & \\
r_{k - 1} & = q_{k + 1} r_k + r_{k + 1}, & 0 & \leq r_{k + 1} < r_k, \\
r_k & = q_{k + 2} r_{k + 1}. & & 
\end{align*}
\end{exercise}
\begin{proof}
From the previous exercise (and the fact that $\gcd(a, b)$ is the least positive integer $d$ such that $(a, b) = (d)$), we get $\gcd(a, b) = \gcd(b, r) = \gcd(r, r_1) = \cdots = \gcd(r_k, r_{k + 1})$.
\\
\\
The process eventually stops since $b, r, r_1, r_2, \hdots$ is a strictly decreasing sequence of non-negative integers (in fact, the sequence stops once it reaches zero, in this case $r_{k + 2} = 0$).  Otherwise, by induction, we can show that
$$r_k \leq r_{k - 1} - 1 \leq \cdots \leq r_1 - (k - 1) \leq r - k \leq b - (k + 1), \qquad k \geq 1$$
When $k = b$, $r_k \leq -1$ cannot be non-negative, a contradiction.
\\
\\
Since $r_{k + 2} = 0$ according to the last division, we have $r_{k + 1} \mid r_k$. Therefore, $\gcd(a, b) = \gcd(r_k, r_{k + 1}) = r_{k + 1}$.
\end{proof}

\newpage

\begin{exercise} \label{c1-ex-3}
Calculate $\gcd(187, 221), \gcd(6188, 4709), \gcd(314, 159)$.
\end{exercise}
\begin{proof}[Solution]
*$\gcd(187, 221) = 17$ since
\begin{align*}
221 & = 1 \cdot 187 + 34 \\
187 & = 5 \cdot 34 + 17 \\
34 & = 2 \cdot 17 + 0
\end{align*}
\\
*$\gcd(6188, 4709) = 17$ since
\begin{align*}
6188 & = 1 \cdot 4709 + 1479 \\
4709 & = 3 \cdot 1479 + 272 \\
1479 & = 5 \cdot 272 + 119 \\
272 & = 2 \cdot 119 + 34 \\
119 & = 3 \cdot 34 + 17 \\
34 & = 2 \cdot 17 + 0
\end{align*}
\\
*$\gcd(314, 159) = 1$
\begin{align*}
314 & = 1 \cdot 159 + 155 \\
159 & = 1 \cdot 155 + 4 \\
155 & = 38 \cdot 4 + 3 \\
4 & = 1 \cdot 3 + 1 \\
3 & = 3 \cdot 1 + 0
\end{align*}
\end{proof}

\newpage

\begin{exercise} \label{c1-ex-4}
Let $d = \gcd(a, b)$. Show how one can use the EucJidean algorithm to find numbers $m$ and $n$ such that $am + bn = d$ (\textit{Hint}: In \hyperref[c1-ex-2]{Exercise \ref*{c1-ex-2}} we have that $d = r_{k + 1}$. Express $r_{k + 1}$ in terms of $r_k$ and $r_{k - 1}$, then in terms of $r_{k - 1}$ and $r_{k - 2}$, etc.)
\end{exercise}
\begin{proof}
Let the following be the Euclidean algorithm for computing $d = \gcd(a, b)$
\begin{align*}
a & = qb + r, & 0 & \leq r < b, \\
b & = q_1 r + r_1, & 0 & \leq r_1 < r, \\
r & = q_2 r_1 + r_2, & 0 & \leq r_2 < r_1, \\
& \enspace\vdots & & \\
r_{k - 1} & = q_{k + 1} r_k + r_{k + 1}, & 0 & \leq r_{k + 1} < r_k, \\
r_k & = q_{k + 2} r_{k + 1}. & & 
\end{align*}
From \hyperref[c1-ex-2]{Exercise \ref*{c1-ex-2}}, we know that $d = r_{k + 1}$. Thus,
\begin{align*}
d & = r_{k + 1} = r_{k - 1} - q_{k + 1} r_k \\
r_k & = r_{k - 2} - q_k r_{k - 1} \\
r_{k - 1} & = r_{k - 3} - q_{k - 1} r_{k - 2} \\
& \enspace\vdots \\
r_2 & = r - q_2 r_1 \\
r_1 & = b - q_1 r \\
r & = a - qb
\end{align*}
Keeping $d$ on one side, and sequentially substituting $r_i$ by a linear combination of $r_{i - 1}$ and $r_{i - 2}$ (according to the above), we get
\begin{align*}
d & = r_{k - 1} - q_{k + 1} r_k & & & & \\
& = r_{k - 1} - q_{k + 1} (r_{k - 2} - q_k r_{k - 1}) & & & & \\
& = m_k r_{k - 2} + n_k r_{k - 1} & (m_k & = -q_{k + 1}, & n_k & = 1 + q_k) \\
& = m_k r_{k - 2} + n_k (r_{k - 3} - q_{k - 1} r_{k - 2}) & & & & \\
& = m_{k - 1} r_{k - 3} + n_{k - 1} r_{k - 2} & (m_{k - 1} & = n_k, & n_{k - 1} & = m_k - n_k q_{k - 1}) \\
& \enspace\vdots & & & & \\
& = m_3 r_1 + n_3 r_2 & & & & \\
& = m_3 r_1 + n_3 (r - q_2 r_1) & & & & \\
& = m_2 r + n_2 r_1 & (m_2 & = n_3, & n_2 & = m_3 - n_3 q_2) \\
& = m_2 r + n_2 (b - q_1 r) & & & & \\
& = m_1 b + n_1 r & (m_1 & = n_2, & n_1 & = m_2 - n_2 q_1) \\
& = m_1 b + n_1 (a - qb) & & & & \\
& = m_0 a + n_0 b & (m_0 & = n_1, & n_0 & = m_1 - n_1 q)
\end{align*}
\end{proof}

\newpage

\begin{exercise}
Find $m$ and $n$ for the pairs $a$ and $b$ given in \hyperref[c1-ex-3]{Exercise \ref*{c1-ex-3}}
\end{exercise}
\begin{proof}[Solution]
*For $\gcd(187, 221) = 17$
\begin{align*}
17 & = 187 - 5 \cdot 34 \\
& = 187 - 5 (221 - 187) \\
& = 6 \cdot 187 - 5 \cdot 221
\end{align*}
\\
*For $\gcd(6188, 4709) = 17$
\begin{align*}
17 & = 119 - 3 \cdot 34 \\
& = 119 - 3 (272 - 2 \cdot 119) \\
& = 7 \cdot 119 - 3 \cdot 272 \\
& = 7 (1479 - 5 \cdot 272) - 3 \cdot 272 \\
& = 7 \cdot 1479 - 38 \cdot 272 \\
& = 7 \cdot 1479 - 38 (4709 - 3 \cdot 1479) \\
& = 121 \cdot 1479 - 38 \cdot 4709 \\
& = 121 (6188 - 4709) - 38 \cdot 4709 \\
& = 121 \cdot 6188 - 159 \cdot 4709
\end{align*}
\\
*For $\gcd(314, 159) = 1$
\begin{align*}
1 & = 4 - 3 \\
& = 4 - (155 - 38 \cdot 4) \\
& = 39 \cdot 4 - 155 \\
& = 39 (159 - 155) - 155 \\
& = 39 \cdot 159 - 40 \cdot 155 \\
& = 39 \cdot 159 - 40 (314 - 159) \\
& = 79 \cdot 159 - 40 \cdot 314
\end{align*}
\end{proof}

\newpage

\begin{exercise} \label{c1-ex-6}
Let $a, b, c \in \mathbb{Z}$. Show that the equation $ax + by = c$ has solutions in integers iff $\gcd(a, b) \mid c$.
\end{exercise}
\begin{proof}
Suppose $ax + by = c$ has solutions in integers, then $c \in (a, b)$. Since $(a, b) = (d)$ where $\gcd(a, b) = d$, we have $c \in (d)$, or simply $\gcd(a, b) \mid c$. Elementarily, note that $a, b$ are divisibly by $\gcd(a, b)$, so $c = ax + by$ is also divisible by $\gcd(a, b)$.
\\
\\
Vice versa, suppose $\gcd(a, b) \mid c$. \hyperref[c1-ex-4]{Exercise \ref*{c1-ex-4}} produces us a pair of integers $(m, n)$ such that $a m + b n = \gcd(a, b)$. Let $k = \frac{c}{\gcd(a, b)}$. By assumption, $k$ is an integer. Hence, $a (km) + b (kn) = k (am + bn) = k \gcd(a, b) = c$, and so $x_0 = km, y_0 = kn$ is a solution to $ax + by = c$.
\end{proof}

\newpage

\begin{exercise} \label{c1-ex-7}
Let $d = \gcd(a, b)$ and $a = da', b = db'$. Show that $\gcd(a', b') = 1$.
\end{exercise}
\begin{proof}[First Proof]
Let $d' = \gcd(a', b')$, then $d' \mid a', b'$. Since $a = da'$ and $b = db'$, we have $d d' \mid a, b$. But $d = \gcd(a, b)$, so any common divisor of $a, b$ must divide $d$, i.e. $d d' \mid d$. Canceling $d$ on both side, we get $d' \mid 1$, or simply $d' = 1$ (since the greatest common divisor is defined to be always positive).
\end{proof}
\begin{proof}[Second Proof]
From \hyperref[c1-ex-6]{Exercise \ref*{c1-ex-6}}, we know there exists some integers $m, n$ such that $am + bn = d$. Dividing $d'$ on both side, we get $a' m + b' n = 1$. Again, by \hyperref[c1-ex-6]{Exercise \ref*{c1-ex-6}}, we know that $\gcd(a', b') \mid 1$. But $\gcd(a', b')$, so we get $\gcd(a', b') = 1$.
\end{proof}

\newpage

\begin{exercise}
Let $x_0$ and $y_0$ be a solution to $ax + by = c$. Show that all solutions have the form $x = x_0 + t \frac{b}{d}, y = y_0 - t \frac{a}{d}$ where $d = (a, b)$ and $t \in \mathbb{Z}$.
\end{exercise}
\begin{proof}
Suppose $(x, y)$ are any solution to $ax + by = c$. Then $a(x - x_0) + b (y - y_0) = 0$, or
$$\frac{b}{d} (y - y_0) = - \frac{a}{d} (x - x_0)$$
Note that $\frac{b}{d} \mid - \frac{a}{d} (x - x_0)$, but $\gcd \left( \frac{a}{d}, \frac{b}{d} \right) = 1$ (\hyperref[c1-ex-7]{Exercise \ref*{c1-ex-7}}), so $\frac{b}{d} \mid x - x_0$. Let $x - x_0 = t \frac{b}{d}$ and substitute into the previous equation, we get
\begin{align*}
\frac{b}{d} (y - y_0) & = - \frac{a}{d} (x - x_0) \\
\frac{b}{d} (y - y_0) & = - \frac{a}{d} \cdot t \cdot \frac{b}{d} \\
y - y_0 & = - t \frac{a}{d}
\end{align*}
Thus any solution is always of the form $x = x_0 + t \frac{b}{d}, y = y_0 - t \frac{a}{d}$ for some $t \in \mathbb{Z}$. On the other hand, for all $t \in \mathbb{Z}$
$$a \left( x_0 + t \frac{b}{d} \right) + b \left( y_0 - t \frac{a}{d} \right) =a x_0 + t \frac{ab}{d} + b y_0 - t \frac{ba}{d} = c$$
So those are all (integer) solutions of $ax + by = c$.
\end{proof}

\newpage

\begin{exercise} \label{c1-ex-9}
Suppose that $u, v \in \mathbb{Z}$ and that $\gcd(u, v) = 1$. If $u \mid n$ and $v \mid n$, show that $uv \mid n$. Show that this is false if $\gcd(u, v) \neq 1$.
\end{exercise}
\begin{proof}
If $\gcd(u, v) \neq 1$, then $u = v = n = 2$ should suffice (as $4 \nmid 2$). As for $\gcd(u, v) = 1$, let $n = ux = vy$ for some $x, y \in \mathbb{Z}$. In particular, we have $v \mid ux$. Yet $\gcd(u, v) = 1$, so we get $v \mid x$. In other words, $uv \mid ux = n$.
\end{proof}

\newpage

\begin{exercise} \label{c1-ex-10}
Suppose that $\gcd(u, v) = 1$. Show that $\gcd(u + v, u - v)$ is either $1$ or $2$.
\end{exercise}
\begin{proof}
Let $d = \gcd(u + v, u - v)$, then $d \mid u + v, u - v$, and so
\begin{align*}
d & \mid (u + v) + (u - v) = 2u \\
d & \mid (u + v) - (u - v) = 2v
\end{align*}
Hence, $d \mid \gcd(2u, 2v)$. Since $\gcd(u, v) = 1$, \hyperref[c1-ex-6]{Exercise \ref*{c1-ex-6}} implies that there exists some $m, n \in \mathbb{Z}$ such that $um + vn = 1$. Multiplying $2$ on both side, we get $(2u)m + (2v)n = 2$, so $\gcd(2u, 2v) \mid 2$. Combining the previous result, we get $d \mid 2$, so either $d = 1$ or $d = 2$.
\end{proof}
\begin{remark}
There is another way to show $\gcd(au, av) = a \gcd(u, v)$, assuming $a \geq 0$. Suppose $d \mid au, av$, let $t = \gcd(d, a)$. By \hyperref[c1-ex-7]{Exercise \ref*{c1-ex-7}}, $\gcd \left( \frac{d}{t}, \frac{a}{t} \right) = 1$, so since $d \mid au$, we get $\frac{d}{t} \mid \frac{a}{t} u$, and thus, $\frac{d}{t} \mid u$. Similarly, $\frac{d}{t} \mid v$, so $\frac{d}{t} \mid \gcd(u, v)$. Multiplying both side by $t$, we get $d \mid t \gcd(u, v) \mid a \gcd(u, v)$. Vice versa, suppose $d \mid a \gcd(u, v)$. Again, let $t = \gcd(a, d)$, then we get $\frac{d}{t} \mid \frac{a}{t} \gcd(u, v)$, so $\frac{d}{t} \mid \gcd(u, v)$. In particular, $\frac{d}{t} \mid u$, so by multiplying $t$ on both side, we get $d \mid tu \mid au$. Similarly, $d \mid av$.
\\
\\
What we have just shown is that $d \mid au, av$ iff $d \mid a \gcd(u, v)$. But $d \mid au, av$ iff $d \mid \gcd(au, av)$, so $d \mid \gcd(au, av)$ iff $d \mid a \gcd(u, v)$. Thus,
$$\gcd(au, av) = |\gcd(au, av)| = |a \gcd(u, v)| = a \gcd (u, v)$$
\end{remark}

\newpage

\begin{exercise}
Show that $\gcd(a, a + k) \mid k$
\end{exercise}
\begin{proof}
Like previous exercise, let $d = \gcd(a, a + k)$. Then $d \mid a, a + k$, so $d \mid (a + k) - a = k$.
\end{proof}

\newpage

\begin{exercise}
Suppose that we take several copies of a regular polygon and try to fit them evenly about a common vertex. Prove that the only possibilities are six equilateral triangles, four squares, and three hexagons.
\end{exercise}
\begin{proof}
Since the vertex angle of an regular $n$-gon is $\frac{n - 2}{n} \pi$, so if there are $k$ copies that can fit around a vertex, then we have
\begin{align*}
k \left( \frac{n - 2}{n} \pi \right) & = 2 \pi \\
k (n - 2) = 2n
\end{align*}
Necessarily, $(n - 2) \mid 2n = 2(n - 2) + 4$, so we get $(n - 2) \mid 4$. Note that $n - 2 \geq 1$, so the only possibilities are $n - 2 \in \{ 1, 2, 4 \}$. Equivalently, $n \in \{ 3, 4, 6 \}$.
\begin{enumerate}
	\item If $n = 3$: then $k = 2n/(n - 2) = 6$. So we can fit 6 equilateral triangles around a vertex.
	\item If $n = 4$: then $k = 2n/(n - 2) = 4$. So we can fit 4 squares around a vertex.
	\item If $n = 6$: then $k = 2n/(n - 2) = 3$. So we can fit 3 regular hexagons around a vertex.
\end{enumerate}
\end{proof}

\newpage

\begin{exercise} \label{c1-ex-13}
Let $n_1, n_2, \hdots, n_s \in \mathbb{Z}$. Define the greatest common divisor $d$ of $n_1, n_2, \hdots, n_s$ and prove that there exist integers $m_1, m_2, \hdots, m_s$ such that $n_1 m_1 + n_2 m_2 + \cdots + n_s m_s = d$.
\end{exercise}
\begin{proof}
Let $(n_1, n_2, \hdots, n_s) = \{ z_1 n_1 + z_2 n_2 + \cdots + z_s n_s : z_1, z_2, \hdots, z_s \in \mathbb{Z} \}$. Since $|n_1| \in (n_1, n_2, \hdots, n_s)$, there exists some positive integers in $(n_1, n_2, \hdots, n_s)$. By well-ordering principle, let $d'$ to be the smalllest such. We will show that $d = d'$, and hence, there exists integers $m_1, m_2, \hdots, m_s$ such that $n_1 m_1 + n_2 m_2 + \cdots + n_s m_s = d$.
\\
\\
Since $d = \gcd(n_1, n_2, \hdots, n_s)$, then given $d' = n_1 m_1 + n_2 m_2 + \cdots + n_s m_s$, we must have $d \mid d'$. On the other hand, $d'$ must divide each $n_i$ ($i = \overline{1, s}$). In other words, $d'$ is a common divisor, so by definition, $d' \mid d$. Both $d$ and $d'$ are positive, so we must have $d = d'$.
\end{proof}

\newpage

\begin{exercise}
Discuss the solvability of $a_1 x_1 + a_2 x_2 + \cdots + a_r x_r = c$ in integers (\textit{Hint}: Use \hyperref[c1-ex-13]{Exercise \ref*{c1-ex-13}} to extend the reasoning behind \hyperref[c1-ex-6]{Exercise \ref*{c1-ex-6}}.)
\end{exercise}
\begin{proof}[Solution]
Suppose there exists some $x_1, x_2, \hdots, x_r$ such that $a_1 x_1 + a_2 x_2 + \cdots + a_r x_r = c$, then $\gcd(a_1, a_2, \hdots, a_r) \mid c$ since $\gcd(a_1, a_2, \hdots, a_r)$ divides each $a_i$ for $1 \leq i \leq r$. Vice versa, suppose $\gcd(a_1, a_2, \hdots, a_r) \mid c$. Then by \hyperref[c1-ex-13]{Exercise \ref*{c1-ex-13}}, there exists some integers $m_1, m_2, \hdots, m_r$ such that $a_1 m_1 + a_2 m_2 + \cdots + a_r m_r = \gcd(a_1, a_2, \hdots, a_r)$. Denote $k = c/\gcd(a_1, a_2, \hdots, a_r)$, which is an integer by assumption. We thus get
\begin{align*}
a_1 (k m_1) + a_2 (k m_2) + \cdots + a_r (k m_r) & = k (a_1 m_1 + a_2 m_2 \cdots + a_r m_r) \\
& = k \gcd(a_1, a_2, \hdots, a_r) = c
\end{align*}
In other words, there exists a solution to $a_1 x_1 + a_2 x_2 + \cdots + a_r x_r = c$
\end{proof}
\begin{remark}
One can solve such equation as follows:
\begin{enumerate}
	\item We group the first $r - 1$ variables so that $(a_1 x_1 + \cdots + a_{r - 1} x_{r - 1}) + a_r x_r = c$.
	\item For the first group $a_1 x_1 + \cdots + a_{r - 1} x_{r - 1}$, it is always a multiple of $d = \gcd(a_1, \hdots, a_{r - 1})$ and vice versa, so we can set it to $d y$ for some integer $y$.
	\item Since $\gcd(a_1, \hdots, a_r) = \gcd(\gcd(a_1, \hdots, a_{r - 1}), a_r) = \gcd(d, a_r)$ and $\gcd(a_1, \hdots, a_r) \mid c$, there exists a solution to $dy + a_r x_t = c$.
	\item Fixing $y$, we repeat the process for $a_1 x_1 + \cdots + a_{r - 1} x_{r - 1} = dy$. Note that $d = \gcd(a_1, \hdots, a_{r - 1})$ so there exists a solution to $a_1 x_1 + \cdots + a_{r - 1} x_{r - 1} = d$. Scaling by $y$ gives us a solution.
\end{enumerate}
However, there is no closed form for the solutions.
\end{remark}

\newpage

\begin{exercise} \label{c1-ex-15}
Prove that $a \in \mathbb{Z}$ is the square of another integer iff $\operatorname{ord}_p a$ is even for all primes $p$. Give a generalization.
\end{exercise}
\begin{proof}
We prove that $a > 0$ is an $r$-th power of some integer iff $\operatorname{ord}_p a$ is divisible by $r$ for all primes $p$. Suppose $a = b^r$ for some $b \in \mathbb{Z}$. Then $\operatorname{ord}_p a = \operatorname{ord}_p b^r = r \operatorname{ord}_p b$ by unique factorization. Thus, $r \mid \operatorname{ord}_p a$ for each prime $p$. Vice versa, suppose $r \mid \operatorname{ord}_p a$. Let $\beta_p = \frac{\operatorname{ord}_p a}{r}$, and $b = \prod_p p^{\beta_p}$ (the sign of $b$ is so that $a$ has the same sign as $b^r$). We then have $b^r = \prod_p p^{r \beta_p} = \prod_p p^{\operatorname{ord}_p a} = a$.
\end{proof}

\newpage

\begin{exercise} \label{c1-ex-16}
If $\gcd(u, v) = 1$ and $uv = a^2$ (with $u, v \geq 0$), show that both $u$ and $v$ are squares.
\end{exercise}
\begin{proof}[First proof]
Let $b = \gcd(u, a)$ and $c = \gcd(v, a)$. We will show that $u = b^2$ and $v = c^2$. By \hyperref[c1-ex-4]{Exercise \ref*{c1-ex-4}}, there exists some $x, y \in \mathbb{Z}$ such that $b = ux + ay$. Squaring on both side, we get
\begin{align*}
b^2 & = u^2 x^2 + 2uaxy + a^2 y^2 \\
& = u^2 x^2 + 2uaxy + uvy^2 \\
& = u(ux^2 + 2axy + vy^2) \\
\end{align*}
Hence, $u \mid b^2$. On the other hand, there also exists some $m,n \in \mathbb{Z}$ such that $um + vn = 1$. Multiplying $u$ on both side we get $u = u^2 m + uv n = u^2 m + a^2 n$. Note that $b = \gcd(u, a)$, so $b^2 \mid u^2, a^2$. From there, we have $b^2 \mid u^2 m + a^2 n = u$. We conclude that $u = b^2$ (both are non-negative). Similarly, $v = c^2$.
\end{proof}
\begin{proof}[Second proof]
Let $u = \prod_p p^{\mu_p}, v = \prod_p p^{\nu_p}$ be the factorizations of $u, v$. Since $\gcd(u, v) = 1$, we must have $\mu_p \nu_p = 0$ at each prime $p$ (i.e. the exponent of $p$ in $u, v$ cannot both be non-zero). Yet $uv = a^2$, so $\mu_p + \nu_p$ must be divisible by $2$. Since one of them is zero, it must be that the other must be divisible $2$. In either case, $\mu_p, \nu_p$ are always even for each prime $p$. By \hyperref[c1-ex-15]{Exercise \ref*{c1-ex-15}}, $u, v$ must be squares.
\end{proof}

\newpage

\begin{exercise}
Prove that the square root of $2$ is irrational, i.e., that there is no rational number $r = a/b$ such that $r^2 = 2$.
\end{exercise}
\begin{proof}
Suppose otherwise, that $a/b$ is a rational number such that $(a/b)^2 = 2$. Equivalently, $a^2 = 2b^2$. Consider the exponent of the prime factor $2$ on both side, we have $2 \operatorname{ord}_2 a = 1 + \operatorname{ord}_2 b$, which is impossible since the left-hand side (LHS) is even, while the right-hand side (RHS) is odd.
\end{proof}
\begin{exercise}
Prove that $\sqrt[n]{m}$ is irrational if $m$ is not the $n$-th power of an integer.
\end{exercise}
\begin{proof}
Suppose otherwise, that $a/b$ is a rational number such that $(a/b)^n = m$. Equivalently, $a^n = m b^n$. Since $m$ is not the $n$-th power of an integer, \hyperref[c1-ex-15]{Exercise \ref*{c1-ex-15}} (or rather the generalization) tells us that, for some prime $p$, $\operatorname{ord}_p m$ is not divisible by $n$. Consider the exponent of $p$ on both side of $a^n = m b^n$, we must have $n \operatorname{ord}_p a = \operatorname{ord}_p m + n \operatorname{ord}_p b$. In other words, $n (\operatorname{ord}_p a - \operatorname{ord}_p b) = \operatorname{ord}_p m$, so $\operatorname{ord}_p m$ must be divisible by $n$, contradicting the previous assertion. Therefore, $\sqrt[n]{m}$ is irrational.
\end{proof}

\newpage

\begin{exercise}
Define the least common multiple of two integers $a$ and $b$ to be an integer $m$ such that $a \mid m, b \mid m$, and $m$  divides every common multiple of $a$ and $b$. Show that such an $m$ exists. It is determined up to sign. We shall denote it by $\operatorname{lcm}(a, b)$ (or $[a, b]$).
\end{exercise}
\begin{proof}
Let $P$ be the intersection of $a \mathbb{Z} = \{ an : n \in \mathbb{Z} \}$ and $b \mathbb{Z} = \{ bn : n \in \mathbb{Z} \}$. Note that $|ab| \in P$, so there exists some positive integer in $P$. By well-ordering principle, there exists the smallest such $m > 0$. Since $m \in a \mathbb{Z}$, $m$ is a multiple of $a$. Similarly, $m$ is a multiple of $b$.
\\
\\
To show $m$ is the least common multiple of $a$ and $b$, we need to show $m \mid \ell$ for any $\ell$ that is a common multiple of both $a$ and $b$. Let $m = q \ell + r, 0 \leq r < \ell$ be the Euclidean division of $m$ by $\ell$. Since $a \mid m, \ell$, we also have $a \mid m - q \ell = r$. Similarly, $b \mid r$. However, $m$ is the least positive integer that is a multiple of both $a$ and $b$, so $r = 0$ necessarily. We conclude that $m \mid \ell$.
\\
\\
Now if $\ell \mid m$ additionally, then we must have $|\ell| = |m|$, i.e. the least common multiple is uniquely defined up to its sign.
\end{proof}

\newpage

\begin{exercise}
Prove the following
\begin{enumerate}
	\item $\operatorname{ord}_p \operatorname{lcm}(a, b) = \max(\operatorname{ord}_p a, \operatorname{ord}_p b)$
	\item $\gcd(a, b) \operatorname{lcm}(a, b) = ab$ (assuming $a, b$ are non-negative)
	\item $\gcd(a + b, \operatorname{lcm}(a, b)) = \gcd(a, b)$
\end{enumerate}
\end{exercise}
\begin{proof}
\underline{Part 1:} let $m = \operatorname{lcm}(a, b)$, then $a \mid m$. Using factorization, this means that $p^{\operatorname{ord}_p a} \mid p^{\operatorname{ord}_p m}$, or simply $\operatorname{ord}_p a \leq \operatorname{ord}_p m$ for each prime $p$. Similarly, $\operatorname{ord}_p b \leq \operatorname{ord}_p m$, so $\max(\operatorname{ord}_p a, \operatorname{ord}_p b) \leq \operatorname{ord}_p m$. On the other hand, the number $\ell = \prod_p p^{\max(\operatorname{ord}_p a, \operatorname{ord}_p b)}$ is a common multiple of $a$ and $b$ (note there are finite many prime with positive exponent) since $\operatorname{ord}_p \ell \geq \operatorname{ord}_p a, \operatorname{ord}_p b$ by definition, so $m \mid \ell$. In particular, $\operatorname{ord}_p m \leq \operatorname{ord}_p \ell = \max(\operatorname{ord}_p a, \operatorname{ord}_p b)$. Hence, $\operatorname{ord}_p m = \max(\operatorname{ord}_p a, \operatorname{ord}_p b)$.
\\
\underline{Part 2:} by similar argument, we also have $\operatorname{ord}_p \gcd(a, b) = \min(\operatorname{ord}_p a, \operatorname{ord}_p b)$ for each prime $p$. Using the identity $\max(u, v) + \min(u, v) = u + v$ (wlog, assume $u \geq v$, then $\max(u, v) + \min(u, v) = u  + v$), we get
\begin{align*}
\operatorname{ord}_p (ab) & = \operatorname{ord}_p a + \operatorname{ord}_p b \\
& = \min(\operatorname{ord}_p a, \operatorname{ord}_p b) + \max(\operatorname{ord}_p a, \operatorname{ord}_p b) \\
& = \operatorname{ord}_p \gcd(a, b) + \operatorname{ord}_p \operatorname{lcm}(a, b)
\end{align*}
As $p$ prime varies, this means $\gcd(a, b) \operatorname{lcm}(a, b) = ab$ (as $a, b \geq 0$).
\\
\underline{Part 3:} we first show the distribution law $\gcd(x, \operatorname{lcm}(y, z)) = \operatorname{lcm} (\gcd(x, y), \gcd(x, z))$. Indeed, at each prime $p$
\begin{align*}
\operatorname{ord}_p \gcd(x, \operatorname{lcm}(y, z)) & = \min(\operatorname{ord}_p x, \max(\operatorname{ord}_p y, z)) \\
& = \max(\min(\operatorname{ord}_p x, \operatorname{ord}_p y), \min(\operatorname{ord}_p x, \operatorname{ord}_p z)) \\
& = \operatorname{lcm} (\gcd(x, y), \gcd(x, z))
\end{align*}
The second-to-last equality is due to distribution law on maximum and minimum. From there, we have
\begin{align*}
\gcd(a + b, \operatorname{lcm}(a, b)) & = \operatorname{lcm}(\gcd(a + b, a), \gcd(a + b, b)) \\
& = \operatorname{lcm}(\gcd(b, a), \gcd(a, b)) = \gcd(a, b)
\end{align*}
The second-to-last equality equality is due to \hyperref[c1-ex-1]{Exercise \ref*{c1-ex-1}}.
\end{proof}

\newpage

\begin{exercise}
Prove that $\operatorname{ord}_p (a + b) \geq \min(\operatorname{ord}_p a, \operatorname{ord}_p b)$ with equality holding if $\operatorname{ord}_p a \neq \operatorname{ord}_p b$.
\end{exercise}
\begin{proof}
Let $u = \operatorname{ord}_p a, v = \operatorname{ord}_p b$, then $a = p^u m, b = p^v n$ for some $p \nmid m, n$. Without loss of generality (WLOG), suppose $u \geq v$. We then have $a + b = p^v (p^{u - v} m + n)$, so $\operatorname{ord}_p (a + b) \geq v = \min(u, v) = \min(\operatorname{ord}_p a, \operatorname{ord}_p b)$. If $u \neq v$, then $p^{u - v} m$ is divisible by $p$, but $n$ is not, so $p^{u - v} m + n$ is not divisible by $p$. Hence, $\operatorname{ord}_p (a + b) = v$, i.e. equality holds.
\end{proof}

\newpage

\begin{exercise}
Almost alI the previous exercises remain valid if instead of the ring $\mathbb{Z}$ we consider the ring $k[x]$. Indeed, in most we can consider any EucJidean domain. Convince yourself of this fact. For simplicity we shall continue to work in $\mathbb{Z}$
\end{exercise}
\begin{proof}
Recall that a Euclidean domain $D$ is an integral domain with a function $\lambda: D \setminus \{ 0 \} \to \mathbb{Z}_{\geq 0}$ (non-negative integer) such that if $a, b \in R, b \neq 0$, there exists $q, r \in R$ such that $a = qb + r$, and either $r = 0$ or $\lambda(r) < \lambda(b)$ (the function is not required to be compatible with other structure on $D$).
\begin{enumerate}
	\item Exercise 3, 5, 12, 17, 18 are explicitly stated in $\mathbb{Z}$, so there's no generalization to Euclidean domain.
	\item Exercise 1, 2, 4 depends directly on the fact that $D$ is Euclidean domain.
	\item Exercise 6, 7, 8, 9, 11, 13, 14 rely on the fact that $D$ is a PID (or rather a Bézout domain, where every finitely generated ideal is principal, not \emph{every} ideal is principal).
	\item Exercise 10 (generalization) works in any commutative ring with unity: if $(u, v) = (1) = D$, then $(2) \subseteq (u + v, u - v)$ (thus if $(d) = (u + v, u - v)$, then $d \mid 2$).
	\\
	\\
	The proof is as follows: given $r \in D$, then there exists some $m, n \in D$ such that $um + vn = r$. We then get
	\begin{align*}
	2r & = (2u) m + (2v) n = [(u + v) + (u - v)] m + [(u + v) - (u - v)] n \\
	& = (u + v) (m + n) + (u - v) (m - n)
	\end{align*}
	Therefore, $2r \in (u + v, u - v)$.
	\item Exercise 16 (generalization) works in any commutative ring $R$ with unity: suppose $I, J, K$ are ideals such that $IJ = K^2$ and $I + J = R$, then $I = (I + K)^2, J = (J + K)^2$.
	\\
	\\
	The proof is as follows:
	\begin{enumerate}
		\item Suppose $i \in I$, then there exists $j \in J$ such that $i + j = 1$. Thus, $i = i^2 + ij$. But $ij \in IJ = K^2$, so we can represent $ij = \sum_\alpha k_\alpha l_\alpha$ for some $k_\alpha, l_\alpha \in K$. In other words, $i = i^2 + \sum_\alpha k_\alpha l_\alpha$. Since $I, K \subseteq I + K$, we know that $i \in I^2 + K^2 \subseteq (I + K)^2$.
		\item Vice versa, note that an element of $(I + K)^2$ is always of the form
		\begin{align*}
		\sum_\alpha (i_\alpha + k_\alpha) (j_\alpha + l_\alpha) & = \sum_\alpha i_\alpha j_\alpha + \sum_\alpha (i_\alpha l_\alpha + j_\alpha k_\alpha) + \sum_\alpha k_\alpha l_\alpha
		\end{align*}
		for some $i_\alpha, j_\alpha \in I; k_\alpha, l_\alpha \in K$. In the latter sum (with 3 terms): the first term is in $I^2 \subseteq I$, the second term is in $IK \subseteq I$, and the last term is in $K^2 = IJ \subseteq I$. Thus, $(I + K)^2 \subseteq I^2 + IK + K^2 \subseteq I + I + I = I$.
	\end{enumerate}
	Since we have both inclusion of $I$ and $(I + K)^2$ into one another, we get $I = (I + K)^2$. Similarly, $J = (J + K)^2$.
	\\
	\\
	When $R = \mathbb{Z}$ and $I = (u), J = (v), K = (a)$ (with $IJ = (uv), K^2 = (a^2), I + J = (\gcd(u, v)) = (1)$), we get the original exercise.
	\item Exercise 15, 19, 20a, 21 work in a UFD.
	\item However, 20b and 20c need a modification (both works in Bézout domain)
	\begin{enumerate}
		\item Part 20b: show that $(a, b) \cdot [(a) \cap (b)] = (ab)$.
		\\
		\\
		To prove, let $(d) = (a, b), (m) = (a) \cap (b)$. Then $\frac{ab}{d} = a \frac{b}{d}$, so $\frac{ab}{d} \in (a)$. Similarly, $\frac{ab}{d} \in (b)$, so $\frac{ab}{d} \in (m)$ necessarily. In other words, $ab \in (dm)$. On the other hand, suppose $d = ax + by$ and $m = az = bw$ for some $x, y, z, w \in D$. Then $dm = (ax)(bw) + (by)(az) = ab(xw + yz)$, so $dm \in (ab)$. We conclude that $(ab) = (dm) = (d) (m) = (a, b) \cdot [(a) \cap (b)]$.
		\item Part 20c: $(a + b) + [(a) \cap (b)] = (a, b)$.
		\\
		\\
		To prove, we need the identity $(x) + [(y) \cap (z)] = (x, y) \cap (x, z)$. Note if $I = (p_i : 1 \leq i \leq n)$ and $J = (q_j : 1 \leq j \leq m)$ are finitely generated ideals, then $IJ = (p_i q_j : 1 \leq i \leq n, 1 \leq j \leq m)$, so 
		\begin{align*}
		(x, y) (y, z) (z, x) & = (x^2 y, x^2 z, y^2 z, y^2 x, z^2 x, z^2 y, xyz) \\
		& = (x, y, z) (xy, yz, zx) \\
		& = (x, y, z) [(xy, zx) + (yz)] \\
		& = (x, y, z) [(x)(y, z) + (yz)]
		\end{align*}
		Yet by part (20b) (and the fact that $D$ is a PID), we have $(x, y) (z, x) = [(x, y) + (z, x)] [(x, y) \cap (z, x)] = (x, y, z) [(x, y) \cap (z, x)]$ and $(yz) = (y, z) [(y) \cap (z)]$. Therefore,
		\begin{align*}
		(x, y, z) [(x, y) \cap (z, x)] (y, z) & = (x, y) (y, z) (z, x) \\
		& = (x, y, z) [(x)(y, z) + (yz)] \\
		& = (x, y, z) [(x)(y, z) + (y, z)[(y) \cap (z)]] \\
		& = (x, y, z) (y, z) [(x) + [(y) \cap (z)]]
		\end{align*}
		We can cancel ideals in a PID, which helps us get the identity $(x, y) \cap (z, x) = (x) + [(y) \cap (z)]$. If one to go by normal route, let $I = (x, y, z) (y, z) = (q) \neq (0)$, then $(q) [(x, y) \cap (z, x)] = (q) [(x) + [(y) \cap (z)]]$. If $a \in (x, y) \cap (z, x)$, then there exists some $v \in D, b \in (x) + [(y) \cap (z)]$ such that $qa = (qv) b$. In other words, $a = bv$ since $q \neq 0$, so $a \in (x) + [(y) \cap (z)]$. Similarly, we also have $a \in (x, y) \cap (z, x)$ whenever $a \in (x) + [(y) \cap (z)]$.
		\\
		\\
		Using the identity above, we get $(a + b) \cap [(a) \cap (b)] = (a + b, a) \cap (a + b, b) = (b, a) \cap (a, b) = (a, b)$.
	\end{enumerate}
\end{enumerate}
\end{proof}

\newpage

\begin{exercise}
Suppose that $a^2 + b^2 = c^2$ with $a, b, c \in \mathbb{Z}$. For example, $3^2 + 4^2 = 5^2$ and $5^2 + 12^2 = 13^2$. Assume that $\gcd(a, b) = \gcd(b, c) = \gcd(c, a) = 1$. Prove that there exist integers $u$ and $v$ such that $c - b = 2u^2$ and $c + b = 2v^2$ and $\gcd(u, v) = 1$ (there is no loss in
generality in assuming that $b$ and $c$ are odd and that $a$ is even). Consequently $a = 2uv$, $b = v^2 - u^2$, and $c = v^2 + u^2$. Conversely show that if $u$ and $v$ are given, then the three numbers $a$, $b$, and $c$ given by these formulas satisfy $a^2 + b^2 = c^2$.
\end{exercise}
\begin{proof}
The converse direction is rather apparent: $(2uv)^2 + (v^2 - u^2)^2 = 4u^2 v^2 + v^4 - 2 v^2 u^2 + u^4 = v^4 + 2 v^2 u^2 + u^4 = (v^2 + u^2)^2$. As for the forward direction, note if $a, b$ are both even, then $c$ is also even, contradicting the pairwise co-prime condition. If $a, b$ are both odd, then considering modulo $4$: $c^2 \equiv 0, 1 \pmod{4}$, yet $a^2 + b^2 \equiv 1 + 1 \equiv 2 \pmod{4}$ (since $a, b$ are odd), contradicting $a^2 + b^2 = c^2$. Therefore, WLOG, we can assume that $a$ is even, and $b$ is odd, which makes $c$ also odd.
\\
\\
Rewrite $a^2 = c^2 - b^2 = (c - b)(c + b)$. By \hyperref[c1-ex-10]{Exercise \ref*{c1-ex-10}}, either $\gcd(c - b, c + b) = 1$, or $\gcd(c - b, c + b) = 2$. Both $b, c$ are odd, so $c - b, c + b$ are even, thus $\gcd(c - b, c + b) = 2$. Dividing $4$ on both side, we get
$$\left( \frac{a}{2} \right)^2 = \frac{c - b}{2} \cdot \frac{c + b}{2}$$
Since $\gcd(c - b, c + b) = 2$, we have $\gcd \left( \frac{c - b}{2}, \frac{c + b}{2} \right) = 1$. By \hyperref[c1-ex-16]{Exercise \ref*{c1-ex-16}}, we must have $\frac{c - b}{2} = u^2, \frac{c + b}{2} = v^2$ for some $u, v \in \mathbb{Z}$. In other words, $b = v^2 - u^2, c = v^2 + u^2$, which makes $a = 2uv$.
\end{proof}

\newpage

\begin{exercise} \label{c1-ex-24}
Prove the identities
\begin{enumerate}
	\item $x^n - y^n = (x - y)(x^{n - 1} + x^{n - 2} y + \cdots + y^{n - 1})$.
	\item For $n$ odd, $x^n + y^n = (x + y) (x^{n - 1} - x^{n - 2} y + x^{n - 3} y^2 - \cdots + y^{n - 1})$.
\end{enumerate}
\end{exercise}
\begin{proof}
\underline{Part 1:}
\begin{align*}
(x - y) \sum_{k = 0}^{n - 1} x^k y^{n - 1 - k} & = \sum_{k = 0}^{n - 1} x^{k + 1} y^{n - 1 - k} - \sum_{k = 0}^{n - 1} x^k y^{n - k} \\
& = x^n + \sum_{k = 0}^{n - 2} x^{k + 1} y^{n - (k + 1)} - \left[ \sum_{k = 1}^{n - 1} x^k y^{n - k} + y^n \right] \\
& = x^n + \sum_{k = 1}^{n - 1} x^k y^{n - k} - \sum_{k = 1}^{n - 1} x^k y^{n - k} - y^n \\
& = x^n - y^n
\end{align*}
\underline{Part 2:}
\begin{align*}
(x + y) \sum_{k = 0}^{n - 1} (-1)^k x^k y^{n - 1 - k} & = \sum_{k = 0}^{n - 1} (-1)^k x^{k + 1} y^{n - 1 - k} + \sum_{k = 0}^{n - 1} (-1)^k x^k y^{n - k} \\
& = x^n - \sum_{k = 0}^{n - 2} (-1)^{k + 1} x^{k + 1} y^{n - (k + 1)} + \left[ \sum_{k = 1}^{n - 1} (-1)^k x^k y^{n - k} + y^n \right] \\
& = x^n + \sum_{k = 1}^{n - 1} (-1)^k x^k y^{n - k} + \sum_{k = 1}^{n - 1} (-1)^k x^k y^{n - k} + y^n \\
& = x^n + y^n
\end{align*}
\end{proof}

\newpage

\begin{exercise}
If $a^n - 1$ is a prime for $a, n > 1$, show that $a = 2$ and that $n$ is a prime. Primes of the form $2^p - 1$ are called \emph{Mersenne primes}. For example, $2^3 - 1 = 7$ and $2^5 - 1 = 31$. It is not known if there are infinitely many Mersenne primes (as of October 2020, there are 50 Mersenne prime found).
\end{exercise}
\begin{proof}
If $a \neq 2$, then $a^n - 1 = (a - 1) (a^{n - 1} + \cdot + 1)$ is a factorization of it (\hyperref[c1-ex-24]{Exercise \ref*{c1-ex-24}}). Since $a \neq 2$, we have $a - 1 \geq 2$. Since $a, n > 1$, $a^{n - 1} + \cdot + 1 > 1$. Thus, $a^n - 1$ cannot be a prime by definition.
\\
\\
If $n$ is not a prime, then let $n = uv$ be a factorization with $u, v > 1$. We then have $a^n - 1 = (a^u)^v - 1 = (a^u - 1) (a^{u(v - 1)} + \cdots + 1)$ (\hyperref[c1-ex-24]{Exercise \ref*{c1-ex-24}}. Since $a, u > 1$, $a^u - 1 > 1$. Since $v > 1$, $a^{u(v - 1)} + \cdots + 1 > 1$. Hence, $a^n - 1$ cannot be a prime by definition.
\\
\\
We conclude that $a = 2$ and $n$ is prime, given $a^n - 1$ is a prime.
\end{proof}

\newpage

\begin{exercise}
If $a^n + 1$ is a prime for $a, n > 1$, show that $a$ is even and that $n$ is a power of 2. Primes of the form $2^{2^t} + 1$ are called Fermat primes. For example, $2^{2^1} + 1 = 5$ and $2^{2^2} + 1 = 17$. It is not known if there are infinitely many Fermat primes (as of 2021, there are only 5 Fermat prime found, namely $2^{2^t} + 1$ for $0 \leq t \leq 4$).
\end{exercise}
\begin{proof}
Suppose $a$ is odd, then $a^n$ is also odd, and thus $a^n + 1$ is even. Since $a > 1$, we have $a^n + 1 \geq a + 1 \geq 4$. Thus, $a^n + 1$ cannot be a prime, since the only even prime is $2 < 4$. On the other hand, if $n$ is not a power of $2$, i.e. $n = 2^d m$ for some $m > 1$ odd. By \hyperref[c1-ex-24]{Exercise \ref*{c1-ex-24}}, $a^n + 1 = (a^{2^d})^m + 1 = (a^{2^d} + 1)(a^{2^d (m - 1)} - a^{2^d (m - 2)} + \cdots + 1)$. Again, since $a > 1$, $a^{2^d} + 1 \geq a + 1 \geq 3$. And since $a, m > 1$, $a^{2^d (m - 1)} > a^{2^d (m - 2)} > \cdots > a^{2^d} > 1$, so $a^{2^d (m - 1)} - a^{2^d (m - 2)} + \cdots + 1 = (a^{2^d (m - 1)} - a^{2^d (m - 2)}) + \cdots + (a^{2^d \cdot 2} - a^{2^d}) + 1 > 1$. By definition, $a^n + 1$ cannot be a prime.
\end{proof}

\newpage

\begin{exercise}
For all odd $n$ show that $8 \mid n^2 -1$. If $3 \nmid n$, show that $6 \mid n^2 -1$.
\end{exercise}
\begin{proof}
Suppose $n = 2k + 1, k \in \mathbb{Z}$, then $n^2 - 1 = (2k + 1)^2 - 1 = 4k^2 + 4k = 4 k(k + 1)$. However, one of $k$ or $k + 1$ must be even, so $n^2 - 1$ must be divisible by $4 \cdot 2 = 8$. Suppose $n = 3k \pm 1, k \in \mathbb{Z}$ (since $n \equiv \pm 1 \pmod{3}$ whenever $n$ is not divisible by $3$), then $n^2 - 1 = (3k \pm 1)^2 - 1 = 9k^2 \pm 6k = 3k (3k \pm 2)$. In particular, $3 \mid n^2 - 1$. We have proved $8 \mid n^2 - 1$ (for $n$ odd), so since $\gcd(3, 8) = 1$, we get $24 \mid n^2 - 1$ (\hyperref[c1-ex-9]{Exercise \ref*{c1-ex-9}}).
\end{proof}

\newpage

\begin{exercise}
For all $n$ show that $30 \mid n^5 - n$ and that $42 \mid n^7 - n$.
\end{exercise}
\begin{proof}
*Using the factorization $30 = 2 \cdot 3 \cdot 5$, we prove $n^5 - n$ is divisible by $2, 3, 5$ separately. Note that $n^5 - n = n (n^4 - 1) = n(n^2 - 1)(n^2 + 1) = n (n - 1) (n + 1) (n^2 + 1)$, so since $n - 1, n, n + 1$ are $3$ consecutive integers, one of them must be divisible by $2$ and one of them must be divisible by $3$. As for $5 \mid n^5 - n$, consider the following case
\begin{enumerate}
	\item If $n \equiv 0, \pm 1 \pmod{5}$, then $n (n - 1) (n + 1)$ is divisble by $5$, so $n^5 - n$ is divisible by $5$.
	\item If $n \equiv \pm 2 \pmod{5}$, then $n^2 + 1 \equiv 2^2 + 1 \equiv 0 \pmod{5}$, so $n^5 - n = n(n - 1)(n + 1)(n^2 + 1) \equiv 0 \pmod{5}$.
\end{enumerate}
*Similarly as above, we prove $n^7 - n$ is divisible by $2, 3, 7$ separately. Note that $n^7 - n = n(n^6 - 1) = n(n^3 - 1)(n^3 + 1) = n(n - 1)(n + 1)(n^2 + n + 1)(n^2 - n + 1)$, so $n^7 - n$ is divisble by $2$ and $3$ with the same argument previously. As for $7 \mid n^7 - n$, consider the following case
\begin{enumerate}
	\item If $n \equiv 0, \pm 1 \pmod{7}$, then $n (n - 1) (n + 1)$ is divisble by $7$, so $n^7 - n$ is divisible by $7$.
	\item If $n \equiv 2 \pmod{7}$, then $n^2 + n + 1 \equiv 2^2 + 2 + 1 \equiv 0 \pmod{7}$, so $n^7 - n \equiv 0 \pmod{7}$.
	\item If $n \equiv -2 \pmod{7}$, then $n^2 - n + 1 \equiv (-2)^2 + 2 + 1 \equiv 0 \pmod{7}$, so $n^7 - n \equiv 0 \pmod{7}$.
	\item If $n \equiv 3 \pmod{7}$, then $n^2 - n + 1 \equiv 3^2 - 3 + 1 \equiv 0 \pmod{7}$, so $n^7 - n \equiv 0 \pmod{7}$.
	\item If $n \equiv -3 \pmod{7}$, then $n^2 + n + 1 \equiv (-3)^2 - 3 + 1 \equiv 0 \pmod{7}$, so $n^7 - n \equiv 0 \pmod{7}$.
\end{enumerate}
\end{proof}

\newpage

\begin{exercise} \label{c1-ex-29}
Suppose that $a, b, c, d \in \mathbb{Z}$ and that $\gcd(a, b) = \gcd(c, d) = 1$. If $(a/b) + (c/d) = $ an integer, show that $b = \pm d$.
\end{exercise}
\begin{proof}
Since $(a/b) + (c/d) = (ad + bc)/(bd)$ is an integer, we have $bd \mid ad + bc$. In particular, $d \mid ad + bc$, so $d \mid bc$. Yet $\gcd(c, d) = 1$, so we need $d \mid b$. Similarly, $b \mid d$, so we conclude $|b| = |d|$, or $b = \pm d$ simply.
\end{proof}

\newpage

\begin{exercise}
Prove that $\frac{1}{2} + \frac{1}{3} + \cdots + \frac{1}{n}$ is not an integer.
\end{exercise}
\begin{proof}
Let $2^k \leq n$ be the largest power of $2$ not exceeding $n$. Then it is also the only number $\leq n$ that is divisble by such power $2^k$. Indeed, if $2^k \mid m \leq n$, then $m = 2^k d$ for some $d \geq 1$. If $d \geq 2$, then $n \geq m \geq 2^{k + 1}$, contradicting the maximality of $k$.
\\
\\
We then split the sum into $\frac{1}{2^k}$ and the remaining ones $\frac{p}{q} = \sum_{i \neq 2^k} \frac{1}{i}$ ($\gcd(p, q) = 1$). By the previous remark, each $i$ can be divisible by at most $2^{k - 1}$, so $q$ (which divide $\operatorname{lcm}_{i \neq 2^k} i$) is also divisible by at most $2^{k - 1}$. If we suppose $\frac{1}{2} + \frac{1}{3} + \cdots + \frac{1}{n}$ is an integer, then $\frac{1}{2^k} + \frac{p}{q}$ is that same integer. By \hyperref[c1-ex-29]{Exercise \ref*{c1-ex-29}}, we thus have $q = \pm 2^k$, a contradiction.
\end{proof}

\newpage

\begin{exercise}
Show that $2$ is divisible by $(1 + i)^2$ in $\mathbb{Z}[i]$.
\end{exercise}
\begin{proof}
$(1 + i)^2 = 1 + 2i + i^2 = 2i$, so $-i (1 + i)^2 = 2$, i.e. $(1 + i)^2 \mid 2$.
\end{proof}

\newpage

\begin{exercise}
For $\alpha = a + bi \in \mathbb{Z}[i]$, we defined $\lambda(\alpha) = a^2 + b^2$. From the properties of $\lambda$ deduce the identity $(a^2 + b^2)(c^2 = d^2) = (ac - bd)^2 + (ad + bc)^2$.
\end{exercise}
\begin{proof}
Since $\lambda$ is multiplicative $\lambda((a + bi) (c + di)) = \lambda(a + bi) \lambda(c + di) = (a^2 + b^2)(c^2 + d^2)$. On the other hand, $(a + bi) (c + di) = ac + ad i +  bc i + bd i^2 = (ac - bd) + (ad + bc)i$, so $\lambda((a + bi) (c + di)) = (ac - bd)^2 + (ad + bc)^2$. Comparing the 2 results, we get the desired identity.
\end{proof}
\begin{exercise}
Show that $\alpha \in \mathbb{Z}[i]$ is a unit iff $\lambda(\alpha) = 1$. Deduce that $1$, $-1$, $i$, and $-i$ are the only units in $\mathbb{Z}[i]$.
\end{exercise}
\begin{proof}
If $\alpha \in \mathbb{Z}[i]$ is a unit, then there exists some $\beta \in \mathbb{Z}[i]$ such that $\alpha \beta = 1$. Applying $\lambda$ to both side and use multiplicativity, we get $\lambda(\alpha) \lambda(\beta) = \lambda(1) = 1$. Note that $\lambda(\alpha)$ is always an integer for any Gaussian integer $\alpha$, so we get $\lambda(\alpha) \mid 1$, i.e. $\lambda(\alpha) = 1$ since $\lambda(\alpha) \geq 0$ is a sum of square. Vice versa, suppose $\lambda(\alpha) = 1$. If $\alpha = a + bi$, consider the product of $\alpha$ and its conjugate: $\alpha \overline{\alpha} = (a + bi)(a - bi) = (a^2 + b^2) + (a(-b) + ba) i = \lambda(\alpha) = 1$. By definition, $\alpha$ is a unit.
\\
\\
By the previous result, the units of $\mathbb{Z}[i]$ are of the form $a + bi$ for $a^2 + b^2 = 1$. Since $a^2 \leq a^2 + b^2 = 1$, we have $-1 \leq a \leq 1$.
\begin{enumerate}
	\item If $a = \pm 1$, then $b = 0$. The corresponding units are $\pm 1$.
	\item If $a = 0$, then $b = \pm 1$. The corresponding units are $\pm i$.
\end{enumerate}
\end{proof}

\newpage

\begin{exercise}
Show that $3$ is divisible by $(1 - \omega)^2$ in $\mathbb{Z}[\omega]$.
\end{exercise}
\begin{proof}
Recall that $\omega^2 + \omega + 1 = 0$, so $(1 - \omega)^2 = 1 - 2 \omega + \omega^2 = - 3 \omega$. Also recall that $\omega^3 = 1$, so $3 = 3 \omega^3 = - \omega^2 (1 - \omega)^2$, so $(1 - \omega^2) \mid 3$.
\end{proof}
\begin{exercise}
For $\alpha = a + b \omega \in \mathbb{Z}[\omega]$, we defined $\lambda(\alpha) = a^2 - ab + b^2$. Show that $\alpha$ is a unit iff $\lambda(\alpha) = 1$. Deduce that $\pm 1, \pm \omega, \pm \omega^2$ are the only units in $\mathbb{Z}[\omega]$.
\end{exercise}
\begin{proof}
First we verify that $\overline{\omega} = \omega^2$. Indeed, $\omega^2 = \left( \frac{-1 + \sqrt{-3}}{2} \right)^2 = \frac{1 - 3 - 2 \sqrt{-3}}{4} = \frac{-1 - \sqrt{-3}}{2} = \overline{\omega}$.
\\
\\
Next, we verify $\lambda$ is multiplicative.
\begin{align*}
\lambda(a + b \omega) \lambda(c + d \omega) & = (a^2 - ab + b^2) (c^2 - cd + d^2) \\
& = a^2 c^2 - a^2 cd + a^2 d^2 - ab c^2 + abcd - abd^2 + b^2 c^2 - b^2 cd + b^2 d^2 \\
& = a^2 c^2 + a^2 d^2 + b^2 c^2 + b^2 d^2 + abcd - a^2 cd - b^2 cd - abc^2 - abd^2 \\
\lambda((a + b \omega)(c + d \omega)) & = \lambda(ac + bd \omega^2 + (bc + ad) \omega) \\
& = \lambda((ac - bd) + (bc + ad - bd) \omega) \\
& = (ac - bd)^2 - (ac - bd) (bc + ad - bd) + (bc + ad - bd)^2 \\
& = a^2 c^2 - 2abcd + b^2 d^2 - abc^2 - a^2 cd + abcd + b^2 cd + abd^2 - b^2 d^2 \\
& \qquad + b^2 c^2 + a^2 d^2 + b^2 d^2 + 2abcd - 2 b^2 cd - 2 ab d^2 \\
& = a^2 c^2 + a^2 d^2 + b^2 c^2 + b^2 d^2 + abcd - a^2 cd - b^2 cd - abc^2 - abd^2
\end{align*}
\\
\\
*Now suppose $\alpha = a + b \omega$ is a unit, i.e. $\alpha \beta = 1$ for some $\beta \in \mathbb{Z}[\omega]$. Then $\lambda(\alpha) \lambda(\beta) = \lambda(1) = 1$, so $\lambda(\alpha) \mid 1$ (since $\lambda(\alpha)$ is always an integer). But $\lambda(\alpha) = a^2 - ab + b^2 = \left( a - \frac{b}{2} \right)^2 + \frac{3b^2}{4} \geq 0$, so $\lambda(\alpha) = 1$. On the other hand, if $\lambda(\alpha) = 1$, then note that
\begin{align*}
\alpha \overline{\alpha} & = (a + b \omega)(a + b \overline{\omega}) \\
& = a^2 + b^2 \omega \overline{\omega} + ab (\omega + \overline{\omega}) \\
& = a^2 + b^2 - ab = \lambda(\alpha) = 1
\end{align*}
By definition, $\alpha$ is a unit.
\\
\\
*To find all units of $\mathbb{Z}[\omega]$, we need to solve $a^2 - ab + b^2 = 1$ for $a, b \in \mathbb{Z}$. Rewrite it into $\left( a - \frac{b}{2} \right)^2 + \frac{3b^2}{4} = 1$, then normalize it by multiplying both side by $4$, we get $(2a - b)^2 + 3b^2 = 4$. Since $3b^2$ is a multiple of $3$ that is between $0$ and $4$, we consider the following cases
\begin{enumerate}
	\item $3b^2 = 0$, or $b = 0$: then $(2a - b)^2 = 4a^2 = 4$, so $a = \pm 1$. These correspond to the units $\pm 1$.
	\item $3b^2 = 3$, or $b = \pm 1$: then $(2a - b)^2 = 1$, so $2a - b = \pm 1$.
	\begin{enumerate}
		\item If $b = 1$, then $a \in \{ 0, 1 \}$. These correspond to the units $\omega, 1 + \omega = -\omega^2$.
		\item If $b = -1$, then $a \in \{ 0, -1 \}$.  These correspond to the units $-\omega, -1-\omega = \omega^2$.
	\end{enumerate}
\end{enumerate}
\end{proof}

\newpage

\begin{exercise}
Define $\mathbb{Z}[\sqrt{-2}]$ as the set of complex numbers of the form $a + b \sqrt{-2}$, where $a, b \in \mathbb{Z}$. Show that $\mathbb{Z}[\sqrt{-2}]$ is a ring. Define $\lambda(\alpha) = a^2 + 2b^2$ for $\alpha = a + b \sqrt{-2}$. Use $\lambda$ to show that $\mathbb{Z}[\sqrt{-2}]$ is a Euclidean domain.
\end{exercise}
\begin{proof}
*$\mathbb{Z}[\sqrt{-2}]$ is a ring since
\begin{enumerate}
	\item For any $a + b \sqrt{-2}, c + d \sqrt{-2} \in \mathbb{Z}[\sqrt{-2}]$ (equivalently, $a, b, c, d \in \mathbb{Z}$), we have
	\begin{align*}
	(a + b \sqrt{-2}) + (c + d \sqrt{-2}) & = (a + c) + (b + d) \sqrt{-2} \in \mathbb{Z}[\sqrt{-2}] \\
	(a + b \sqrt{-2}) (c + d \sqrt{-2}) & = (ac - 2bd) + (ad + bc) \sqrt{-2} \in \mathbb{Z}[\sqrt{-2}]
	\end{align*}
	\item Commutativity, associativitiy, and distributivity hold as these are just complex arithmetics.
	\item Identities: since $0 = 0 + 0 \cdot \sqrt{-2}, 1 = 1 + 0 \cdot \sqrt{-2} \in \mathbb{Z}[\sqrt{-2}]$
	\begin{align*}
	(a + b \sqrt{-2}) + (0 + 0 \sqrt{-2}) & = a + b \sqrt{-2} \\
	(a + b \sqrt{-2}) (1 + 0 \sqrt{-2}) & = a + b \sqrt{-2}
	\end{align*}
	\item Additive inverse: $(a + b \sqrt{-2}) + [(-a) + (-b) \sqrt{-2}] = 0$
\end{enumerate}
*To show $\mathbb{Z}[\sqrt{-2}]$, for any $\alpha = a + b \sqrt{-2}$ and $\beta = c + d \sqrt{-2} \neq 0$. Let
$$\frac{\alpha}{\beta} = \frac{(a + b \sqrt{-2})(c - d \sqrt{-2})}{c^2 + 2d^2} = \frac{(ac + 2bd) + (bc - ad) \sqrt{-2}}{c^2 + 2d^2} = r + s \sqrt{-2}$$
for some $r, s \in \mathbb{Q}$. Pick some integer $m, n$ such that $|m - r| \leq 1/2$ and $|n - s| \leq 1/2$ (e.g. $m, n$ are respectively the nearest integers to $r, s$). Let $\rho = m + n \sqrt{-2}$ and $\delta = \alpha - \rho \beta$ (both in $\mathbb{Z}[\sqrt{-2}]$), then $\lambda \left( \frac{\alpha}{\beta} - \rho \right) = (m - r)^2 + 2(n - s)^2 \leq 3/4 < 1$, so $\lambda(\delta) = \lambda(\beta) \lambda \left( \frac{\alpha}{\beta} - \rho \right) < \lambda(\beta)$ (note that $\lambda$ is the square of the absolute value on $\mathbb{C}$). 
\end{proof}
\begin{exercise}
Show that the only units in $\mathbb{Z}[\sqrt{-2}]$ are $1$ and $-1$.
\end{exercise}
\begin{proof}
If $\alpha = a + b \sqrt{-2}$ is a unit, then $\alpha \beta = 1$ for some $\beta \in \mathbb{Z}[\sqrt{-2}]$. Since the absolute value of a complex number is multiplicative, we have $|\alpha| \cdot |\beta| = 1$, or equivalently, $\lambda(\alpha) \lambda(\beta) = |\alpha|^2 |\beta|^2 = 1$. But $\lambda(\alpha) = a^2 + 2b^2$ is a non-negative integer, so we get $\lambda(\alpha) = 1 = a^2 + 2b^2$. In particular, $2b^2 \leq 1$, so $b = 0$ necessarily. From there, we get $a = \pm 1$, so the only \textit{possible} units in $\mathbb{Z}[\sqrt{-2}]$ are $\pm 1$. Checking $1^2 = (-1)^2 = 1$, we conclude that these are the only units.
\end{proof}

\newpage

\begin{exercise}
Suppose that $\pi \in \mathbb{Z}[i]$ and that $\lambda(n) = p$ is a prime in $\mathbb{Z}$. Show that $\pi$ is a prime in $\mathbb{Z}[i]$. Show that the corresponding result holds in $\mathbb{Z}[\omega]$ and $\mathbb{Z}[\sqrt{-2}]$.
\end{exercise}
\begin{proof}
We will show that such holds in any Euclidean domain $D$ with Euclidean function $\lambda: D \to \mathbb{Z}_{\geq 0}$ such that $\lambda$ is multiplicative, and $\alpha$ is a unit whenever $\lambda(\alpha) = 1$. Indeed, suppose $\pi \in D$ such that $\lambda(\pi) = p$ a prime. Then $\pi$ must be irreducible, as whenever $\pi = \alpha \beta$, we have $\lambda(\alpha) \lambda(\beta) = \lambda(\pi) = p$. Since $\lambda(\alpha), \lambda(\beta) \geq 0$, we must have either $\lambda(\alpha) = 1$, or $\lambda(\beta) = 1$. In other words, either $\alpha$ or $\beta$ is a unit.
\\
\\
So $\pi$ is irreducible. We need to show that $\pi$ is prime from this fact. Indeed, an Euclidean domain is always a PID, so whenever $\pi \mid \alpha \beta$, yet $\pi \nmid \alpha$, we let $(\gamma) = (\pi, \alpha)$. Since $(\gamma) \supseteq (\pi)$, we have $\gamma \mid \pi$. $\pi$ is irreducible so either $(\pi, \alpha) = (\gamma) = (\pi)$, or $(\pi, \alpha) = (\gamma) = (1)$. The first case implies that $(\pi) \supseteq (\alpha)$, or simply $\pi \mid \alpha$, contradicting the assumption. Thus, $(\pi, \alpha) = (1)$, and so there exists $x_0, y_0 \in D$ such that $\pi x_0 + \alpha y_0 = 1$. Multiplying both side by $\beta$, we get $\pi (\beta x_0) + (\alpha \beta) y_0 = \beta$. Since $\pi \mid \alpha \beta$, we have $\pi \mid \pi (\beta x_0) + (\alpha \beta) y_0 = \beta$. By definition, $\pi$ is a prime.
\end{proof}

\newpage

\begin{exercise}
Show that in any integral domain a prime element is irreducible.
\end{exercise}
\begin{proof}
Let $p \in R$ be a prime element. Suppose $p = ab$, then $p \cdot 1 = ab$, so $p \mid ab$. By definition, we can assume WLOG that $p \mid a$. Let $a = pc$ for some $c \in R$. Then $p = ab = pcb$, so $1 = cb$ (since $R$ is an integral domain). In other words, $b$ is a unit in $R$. In summary, any factorization $p = ab$ of a prime element has either $a$ or $b$ be a unit. By definition, $p$ must be irreducible.
\end{proof}
\end{document}